\section*{General description of work} 


\subsection*{Background}



A general way to model and quantize uncertainty in an optimization problem is the \emph{Chance Constraints} \cite{geng2019data}. Such kind of constraints allow one to properly incorporate uncertainty into optimization problems and define probability level at which one allows the system to violate constraints. The allowance for original deterministic constraints violation is important to avoid excessive solution conservatism. For example, finding a solution that is robust against each possible realization of uncertainty with unbounded support would result in an empty feasibility set. However, in most cases, chance constraints do not admit a closed-form expression, thus, making it impossible to use them in numerical solvers. These constraints are typically approximated using upper bounds on probability, e.g., Bernstein approximation, \cite{nemirovski2007convex} or using data, i.e., constructing a data-driven approximation. Further, we focus only on the latter type of approximations.

%Paragraph about SA, SAA and other stuff from see comment
%https://www.sciencedirect.com/science/article/abs/pii/S1367578819300306
Chance constrained optimization originated in 1958 by Charnes \cite{charnes1958cost} and since then found a wide range of applications. For example, in economics \cite{yaari1965uncertain}, control theory \cite{calafiore2006scenario}, chemical processes \cite{sahinidis2004optimization} and in machine learning \cite{bertsimas2018robust,caramanis2011robust}.  Chance constraints are generally do not admit a closed-form expression. To this end, several approaches were developed to construct various approximations chance constraints. These approaches further evolved into major research and application fields. Among them are Abmiguous Chance Constraints \cite{nemirovski2012safe, ben2009robust}, Robust Optimization (RO) based Methods \cite{bertsimas2011theory}, Sample Average Approximation (SAA) \cite{sen1992relaxations, ahmed2008solving} and Scenario Approach (Approximation) (SA) \cite{calafiore2005uncertain}. SA and SAA are data-driven approaches, whereas the others are analytical or mixed.


%Paragraph about "despite suck some dick" there are some problems. 
Despite intensive research in the field of Chance Constrained Optimization, each subfield experiences difficulties: analytical approaches suffer from extensive conservativeness, while data-driven ones often are prohibitevely demanding in computational resources for large scale problems. The aforementioned drawbacks are critical in applications that require regular optimization. 
One of such applications is power systems, where a higher amount of renewable generation increases power grid uncertainty, compromises its security, and challenges classical power grid operation and planning policies \cite{koutsoyiannis2016unavoidable}. It is worth mentioning that installation of renewable energy based generation is a common trend and brings substantial amount of randomness into the system \cite{koutsoyiannis2016unavoidable}, \cite{harjanne2019abandoning}. For example, according to recent International Energy Agency (IEA) report \cite{iea2024electricity}, renewable energy generation share growth is up to 50\% and to take 30\% of all conventional electricity generation around the world. 

In power systems applications, the aforementioned drawbacks are crucial. Conservativeness leads to higher power market prices which influences industry and society, computational efficiency must be such that the computations accomplish within the re-evaluation interval defined by local system operator \cite{chen2008probabilistic, koutsoyiannis2016unavoidable, stott2012optimal}.

%Example:
%"Power systems are the backbone of modern society, providing the necessary energy to drive economic activities, industrial processes, and daily life. As the demand for reliable and efficient energy continues to grow, the complexity and significance of power systems have increased, necessitating advanced research and development in this field."

%Example:
%"Power systems are the backbone of modern society, providing the necessary energy to drive economic activities, industrial processes, and daily life. As the demand for reliable and efficient energy continues to grow, the complexity and significance of power systems have increased, necessitating advanced research and development in this field."


\subsection*{Relevance of the work}


The research holds significant importance as it is developing and applying advanced methods such as mirror descent based adaptive importance sampling and A-priori Reduced Scenario Approximation (AR-SA), the research aims to improve the efficiency and accuracy of sample based probability estimation and optimization under uncertainty. These methods contribute to overcoming limitations in current approaches, such as impracticality for real-time operations, overestimation of risks, and computational infeasibility in large-scale problems. The outcomes of this research are expected to enhance the efficiency of commonly used approaches, making them significantly less data demanding. Moreover, the proposed methods and approaches are shown to be effective in power systems applications as they addresses critical challenges in modeling power systems with high amount of installed renewable energy generation, which become ubiquitous.

Therefore, this thesis focuses on improving numerical efficiency of the data-driven approaches. This study leverages advanced statistical methods such as Importance Sampling (IS) for estimation of a Gaussian volume of a polyhedrons complement, which is the reliability estimation of current power system state in power systems with high renewable energy penetration setting. Further, IS is adapted for increasing numerical efficiency of SA for Jointly Chance Constrained linear programs with additive uncertainty and mixed additive-multiplicative uncertainty. The former and latter are met as Joint Chance Constrained Direct Current Optimal Power Flow (JCC DC-OPF) problem and dynamic JCC DC-OPF with Automaged Generation Control (AGC).



\subsection*{Dissertation goals} 



This research aims to develop advanced methods for improving the efficiency and accuracy of reliability assessments and optimization under uncertainty. The results are demonstrated on power systems applications.
To achieve this goal, the following tasks were set up and performed:
% a)	Probability Estimation: Develop numerically efficient methods for estimating the probability of the current power system state being feasible against uncertainties that come from RES generation.
% b)	Single Timestamp Optimization: Develop statistically based alorithms for the construction of data-efficient SA for S-OPF that yield reliable solutions for individual timestamps. Provide theoretical guarantees.
% c)	24-Hour Sequence Optimization: Generalize the results of b) to optimize generator actions throughout a 24-hour period.

\begin{enumerate}
    \item Probability Estimation: Develop numerically efficient methods for estimating the probability measure of a polyhedron's complement. Provide theoretical support by formalizing the convergence theorem for estimate's variance and proving it. Support the theoretical advances by demonstrating algorithm's efficiency on power systems test cases. The power system example is the estimation of the probability that the current power system state is feasible against uncertainties that come from RES generation.
    \item Linear Programming under Additive Uncertainty: Develop statistically based algorithms for the construction of data-efficient Scenario Approximation (SA) for Chance-Constrained Linear Programs that yield reliable solutions, assuming additive Gaussian uncertainty for decision variables. Provide theoretical guarantees that show the improvement for required number of samples for obtaining reliable solution. Demonstrate the approximation method validity numerically on power systems test cases, compare with existing approaches. The power system example is Chance Constrained Optimal Power Flow.
    \item Linear Programming under Mixed Multiplicative-Additive Uncertainty: Generalize results for a non-Gaussian source uncertainty on a non-Gaussian case, consider mixed multiplicative-additive uncertainty. Assuming typical total uncertainty mitigation setup, derive analytical condition for filtering redundant scenarios in this setting. Prove that such scenario filtering increases data efficiency and demonstrate numerically the superiority to the scenario reduction methods and ambiguous chance constraints approach. Provide demonstration on power system example which is sequential time-stamp power system modeling, where major uncertainty contributors in grid are renewable energy sources (photo-voltaic panels, wind farms, hydro electrical generators, stochastic demand) and the algorithm for uncertainty mitigation is linear Automatic Generation Control (AGC).
\end{enumerate}


\newpage

\subsection*{Propositions for defense}

\begin{enumerate}

    \item A numerical iterative method for estimation of a Gaussian volume of a polyhedron's complement. The method is of adaptive importance sampling family, where the sampling distribution is a Gaussian mixture. It iteratively minimizes the variance of the estimate over mixture weights using Mirror Descent. The convexity of the optimization problem is shown, stochastic gradient's expression is derived and, finally, the iterative method's convergence is shown. The method's performance is demonstrated against other estimation algorithms on power systems examples.
    \item An algorithm for construction of Scenario Approximation for linear programming with additive uncertainty is proposed for solving Joint Chance Constrained programs. This algorithm is based on Importance Sampling, where the sampling distribution is Gaussian mixture. The subset of feasibility set is derived based on Gaussianity assumption for the sampling outside of it and constructing the SA based on those scenarios. The theorem on solution reliability and the number of samples required of such importance sampling based SA is stated, the proof was provided. The numerical demonstration is carried out on power systems test cases and compared to classical SA construction algorithms.
    \item An algorithm for construction of Scenario Approximation for linear programming with mixed additive-multiplicative uncertainty is proposed for solving Joint Chance Constrained linear programs. The subset of feasibility set is derived for a-priori elimination of the redundant scenarios. The demonstration is conducted on the power system example which is sequential time-stamp power system modeling, where major uncertainty contributors in grid are renewable energy sources (photo-voltaic panels, wind farms, hydro electrical generators, stochastic demand) and the algorithm for uncertainty mitigation is linear Automatic Generation Control (AGC).
    %The Gaussianity assumption is demostrated to be valid using Shapiro-Wilks statistical tests on real time series. 
    The theorem on solution reliability and the number of samples required for reduced problems is stated, the proof was provided. The numerical demonstration is carried out on power systems test cases and compared to advanced scenario reduction methods and ambiguous chance constrained method.
 
\end{enumerate}


\subsection*{Scientific novelty}
% It seems that statements to defend can very much overlap with the sci novelty, at least in some examples they do.
The scientific novelty is built up from the following results:
\begin{enumerate}
    \item The application of adaptive importance sampling combined with mirror descent methods to power systems, particularly in scenarios involving rare events. The convergence of the algorithm was established through stating and proving a convergence theorem for optimizing estimate's variance. Next, the performance of this novel approach with practical algorithms, such as pmvnorm, was demonstrated on an example from power systems that highlights its effectiveness and potential advantages.
    \item A novel method for constructing scenario approximations, introducing a more efficient and accurate approach to scenario approximation. The theoretical guarantees for this novel construction method were provided, ensuring its mathematical soundness and reliability. Lastly, a numerical demonstrations on power systems examples were conducted to compare the performance of this new scenario approximation method with classical Monte Carlo-based approach, highlighting its efficiency and accuracy.
    \item Introducing an a priori approach to reduce scenario approximations for dynamic optimal power flow with automatic generation control (AGC), enhancing computational efficiency and accuracy, studying normality of generation-demand mismatch using real time series of load, renewable generation of various sources. The statements that ensured validity of the reduction approach and shown an advantage in scenario complexity were proposed. These statements confirm the method's effectiveness and reliability in reducing scenario approximations for chance constrained linear programs with multiplicative uncertainty. A numerical demonstration was conducted on a dynamic optimal power flow problem. The demonstration compares the proposed a-priori approach with other scenario reduction methods, incorporating data-driven distributional robust optimization to showcase its superior performance and practical applicability.
\end{enumerate}

\subsection*{Theoretical and practical significance}
This innovative approach could significantly advance the field of scenario approximation by offering a more efficient and accurate methods. It may lead to advancements in mathematical modeling and optimization techniques. Implementing these new methods could enhance the performance of power system simulations, leading to more accurate predictions and better decision-making in energy management. Providing theoretical guarantees for the new construction method solidifies its mathematical underpinnings, paving the way for its wider acceptance in research and practical applications. Assurance of the method's validity offers confidence to practitioners and decision-makers in utilizing it for scenario approximation in power systems, potentially leading to more reliable system planning and operation. Conducting numerical demonstrations and comparisons contributes to the theoretical understanding of scenario approximation methods, offering insights into their strengths and weaknesses. By comparing the new approach with classical Monte Carlo-based methods, the study can inform practitioners about the performance differences, aiding them in choosing the most suitable method for scenario approximation in power systems. Moreover, numerical experiments included comparison of total execution time with current state-of-the-art methods and show the practical advantages of the proposed methods.
%Classification of quantum phases by quantum means is potentially useful for condensed matter physics. The results on the convergence of VQE are important for the design of quantum optimization algorithms. The proposed method for validating the GHZ state is potentially useful for evaluating the properties of quantum devices in a simple manner.
%Practical significance of the work is supported by the usage of the results in delivering the RFBR grant No.\ 19-31-90159 ``Aspiranty''.



\subsection*{Research methodology}
Methodology included methods of linear algebra, probability theory, mathematical statistics, numerical optimization methods, aspects of optimization methods, software development and models of power systems.
%В работе применялись методы линейной алгебры, теории алгоритмов, а также методы машинного обучения.
\subsection*{The reliability}
The proposed methods and approaches were equipped with theoretical statements that were proven, numerical demonstrations show their validity. Specifically, for iterative methods, the convergence theorem was introduced. The statement introduces a bound for the target estimate's variance and reveals asymptotic behaviour with respect to the number of iterations. For those methods that are applied to Scenario Approximations, a theorem on an improvement in requires number of data samples (scenarios) are introduced. Finally, all the proposed methods are compared with existing methods for solving similar optimization problems using measurable metrics. 

All of the proposed methodologies and approaches were published in WoS, Scopus indexed journals of rank Q1 and presented at reputable international conferences.
%Все предложенные методы и разработанная методика были реализованы и прошли экспериментальную проверку. Предложенные наборыданных были опубликованы.
\subsection*{Validation of the research results}
The main results of the work have been reported in the following scientific conferences and workshops:

\begin{enumerate}
    \item INFORMS Annual Meeting 2021, 1st INFORMS Workshop on Quality, Statistics \& Reliability, October 15, 2021, Indianapolis
    \item Rank A, CDC 2021, 60th IEEE Conference on Decision and Control, December 13-15 2021
    \item Rank A, IEEE PowerTech, Belgrade, June 25-29 2023
    \item Energy Research Seminar, Skoltech, October 12 2021
\end{enumerate}

\subsection*{Personal contribution} All the results of the dissertation were obtained personally by the 
applicant or with his direct involvement. In particular, the applicant performed the search 
and  analysis  of  the  literature  related  to  the  research  topic.  The  applicant  participated in the formulation of aims and objectives of 
the  dissertation  and  developed  experimental  methods.  The  results  of  the  work  were obtained personally by the author or with his direct participation. 

\subsection*{Structure and volume of the dissertation} The dissertation consists of introduction, five chapters, conclusions, bibliography, list of symbols and abbreviations, list of tables, list of figures, list of 113 references. The full 
volume of the dissertation is 105 pages, including 17 figures and 9 tables.