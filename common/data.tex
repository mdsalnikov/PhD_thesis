%%% Основные сведения %%%
\newcommand{\thesisAuthorLastName}{Салников}
\newcommand{\thesisAuthorOtherNames}{Михаил}
\newcommand{\thesisAuthorInitials}{М.}
\newcommand{\thesisAuthorLastNameEn}{Salnikov}
\newcommand{\thesisAuthorOtherNamesEn}{Mikhail}
\newcommand{\thesisAuthorInitialsEn}{M.}

\newcommand{\thesisAuthor}             % Диссертация, ФИО автора
{%
    \texorpdfstring{% \texorpdfstring takes two arguments and uses the first for (La)TeX and the second for pdf
        \thesisAuthorLastName~\thesisAuthorOtherNames% так будет отображаться на титульном листе или в тексте, где будет использоваться переменная
    }{%
        \thesisAuthorLastName, \thesisAuthorOtherNames% эта запись для свойств pdf-файла. В таком виде, если pdf будет обработан программами для сбора библиографических сведений, будет правильно представлена фамилия.
    }
}
\newcommand{\thesisAuthorEn}             % Диссертация, ФИО автора
{%
    \texorpdfstring{% \texorpdfstring takes two arguments and uses the first for (La)TeX and the second for pdf
        \thesisAuthorLastNameEn~\thesisAuthorOtherNamesEn% так будет отображаться на титульном листе или в тексте, где будет использоваться переменная
    }{%
        \thesisAuthorLastNameEn, \thesisAuthorOtherNamesEn% эта запись для свойств pdf-файла. В таком виде, если pdf будет обработан программами для сбора библиографических сведений, будет правильно представлена фамилия.
    }
}

\newcommand{\thesisAuthorShort}        % Диссертация, ФИО автора инициалами
{\thesisAuthorInitials~\thesisAuthorLastName}
\newcommand{\thesisAuthorShortEn}        % Диссертация, ФИО автора инициалами
{\thesisAuthorInitialsEn~\thesisAuthorLastNameEn}

%\newcommand{\thesisUdk}                % Диссертация, УДК
%{\fixme{xxx.xxx}}
\newcommand{\thesisTitle}              % Диссертация, название
{LLM hallucination}
\newcommand{\thesisTitleEn}              % Диссертация, название
{LLM hallucination}
\newcommand{\thesisSpecialtyNumber}    % Диссертация, специальность, номер
{1.2.2}%05.13.18
\newcommand{\thesisSpecialtyTitle}     % Диссертация, специальность, название (название взято с сайта ВАК для примера)
{Математическое моделирование, численные методы и комплексы программ}
\newcommand{\thesisSpecialtyTitleEn}     % Диссертация, специальность, название (название взято с сайта ВАК для примера)
{Mathematical modeling, numerical methods and program complexes}
%% \newcommand{\thesisSpecialtyTwoNumber} % Диссертация, вторая специальность, номер
%% {\fixme{XX.XX.XX}}
%% \newcommand{\thesisSpecialtyTwoTitle}  % Диссертация, вторая специальность, название
%% {\fixme{Теория и~методика физического воспитания, спортивной тренировки,
%% оздоровительной и~адаптивной физической культуры}}
\newcommand{\thesisDegree}             % Диссертация, ученая степень
{кандидата физико-математических наук}
\newcommand{\thesisDegreeEn}             % Диссертация, ученая степень
{Doctor of Philosophy in Physics and Mathematics}
\newcommand{\thesisDegreeShort}        % Диссертация, ученая степень, краткая запись
{канд.~физ.-мат.~наук}
\newcommand{\thesisDegreeShortEn}        % Диссертация, ученая степень, краткая запись
{PhD.~in Phys.-Math.}
\newcommand{\thesisCity}               % Диссертация, город написания диссертации
{Москва}
\newcommand{\thesisCityEn}               % Диссертация, город написания диссертации
{Moscow}
\newcommand{\thesisYear}               % Диссертация, год написания диссертации
{2024}
\newcommand{\thesisOrganization}       % Диссертация, организация
{Автономная некоммерческая образовательная организация высшего образования <<Сколковский институт науки и технологий>>}
\newcommand{\thesisOrganizationNonTitle}       % Диссертация, организация
{Автономная некоммерческая образовательная организация высшего образования <<Сколковский институт науки и технологий>>}
\newcommand{\thesisOrganizationEn}       % Диссертация, организация
{Autonomous Non-Profit Organization for Higher Education\\<<Skolkovo Institute of Science and Technology>>}
\newcommand{\thesisOrganizationEnNonTitle}       % Диссертация, организация
{Autonomous Non-Profit Organization for Higher Education <<Skolkovo Institute of Science and Technology>>}
\newcommand{\thesisOrganizationShort}  % Диссертация, краткое название организации для доклада
{Сколтех}
\newcommand{\thesisOrganizationShortEn}  % Диссертация, краткое название организации для доклада
{Skoltech}

\newcommand{\thesisInOrganization}     % Диссертация, организация в предложном падеже: Работа выполнена в ...
{Автономной некоммерческой образовательной организации высшего профессионального образования <<Сколковский институт науки и технологий>>}

%% \newcommand{\supervisorDead}{}           % Рисовать рамку вокруг фамилии
\newcommand{\supervisorFio}              % Научный руководитель, ФИО
{Елена Николаевна Грязина}
\newcommand{\supervisorRegalia}          % Научный руководитель, регалии
{доктор компьютерных наук, кандидат физико-математических наук}
\newcommand{\supervisorFioShort}         % Научный руководитель, ФИО
{Е.\,Н.~Грязина}
\newcommand{\supervisorRegaliaShort}     % Научный руководитель, регалии
{д.~к.~н, к.ф.-м.н.}

\newcommand{\supervisorFioEn}              % Научный руководитель, ФИО
{Elena Nikolaevna Gryazina}
\newcommand{\supervisorRegaliaEn}          % Научный руководитель, регалии
{Doctor of Computer Sciences, Candidate of Physical and Mathematical Sciences}
\newcommand{\supervisorFioShortEn}         % Научный руководитель, ФИО
{\fixme{E.\,N.~Gryazina}}
\newcommand{\supervisorRegaliaShortEn}     % Научный руководитель, регалии
{\fixme{D.Sc., c.p.-m.s}}

%% \newcommand{\supervisorTwoDead}{}        % Рисовать рамку вокруг фамилии
%% \newcommand{\supervisorTwoFio}           % Второй научный руководитель, ФИО
%% {\fixme{Фамилия Имя Отчество}}
%% \newcommand{\supervisorTwoRegalia}       % Второй научный руководитель, регалии
%% {\fixme{уч. степень, уч. звание}}
%% \newcommand{\supervisorTwoFioShort}      % Второй научный руководитель, ФИО
%% {\fixme{И.\,О.~Фамилия}}
%% \newcommand{\supervisorTwoRegaliaShort}  % Второй научный руководитель, регалии
%% {\fixme{уч.~ст.,~уч.~зв.}}

\newcommand{\opponentOneFio}           % Оппонент 1, ФИО
{\fixme{Фамилия Имя Отчество}}
\newcommand{\opponentOneRegalia}       % Оппонент 1, регалии
{\fixme{доктор физико-математических наук, профессор}}
\newcommand{\opponentOneJobPlace}      % Оппонент 1, место работы
{\fixme{Не очень длинное название для места работы}}
\newcommand{\opponentOneJobPost}       % Оппонент 1, должность
{\fixme{старший научный сотрудник}}

\newcommand{\opponentTwoFio}           % Оппонент 2, ФИО
{\fixme{Фамилия Имя Отчество}}
\newcommand{\opponentTwoRegalia}       % Оппонент 2, регалии
{\fixme{кандидат физико-математических наук}}
\newcommand{\opponentTwoJobPlace}      % Оппонент 2, место работы
{\fixme{Основное место работы c длинным длинным длинным длинным названием}}
\newcommand{\opponentTwoJobPost}       % Оппонент 2, должность
{\fixme{старший научный сотрудник}}

%% \newcommand{\opponentThreeFio}         % Оппонент 3, ФИО
%% {\fixme{Фамилия Имя Отчество}}
%% \newcommand{\opponentThreeRegalia}     % Оппонент 3, регалии
%% {\fixme{кандидат физико-математических наук}}
%% \newcommand{\opponentThreeJobPlace}    % Оппонент 3, место работы
%% {\fixme{Основное место работы c длинным длинным длинным длинным названием}}
%% \newcommand{\opponentThreeJobPost}     % Оппонент 3, должность
%% {\fixme{старший научный сотрудник}}

\newcommand{\opponentOneFioEn}           % Оппонент 1, ФИО
{\fixme{Name name name}}
\newcommand{\opponentOneRegaliaEn}       % Оппонент 1, регалии
{\fixme{Doctor of physcial and mathematical sciences, Professor}}
\newcommand{\opponentOneJobPlaceEn}      % Оппонент 1, место работы
{\fixme{Somewhat long job place name}}
\newcommand{\opponentOneJobPostEn}       % Оппонент 1, должность
{\fixme{Senior research scientist}}

\newcommand{\opponentTwoFioEn}           % Оппонент 2, ФИО
{\fixme{Name name name}}
\newcommand{\opponentTwoRegaliaEn}       % Оппонент 2, регалии
{\fixme{Doctor of physcial and mathematical sciences}}
\newcommand{\opponentTwoJobPlaceEn}      % Оппонент 2, место работы
{\fixme{Main job place with a long long long long long long, reeeeally long title}}
\newcommand{\opponentTwoJobPostEn}       % Оппонент 2, должность
{\fixme{Senior research scientist}}

\newcommand{\leadingOrganizationTitle} % Ведущая организация, дополнительные строки. Удалить, чтобы не отображать в автореферате
{...}
\newcommand{\leadingOrganizationTitleEn} % Ведущая организация, дополнительные строки. Удалить, чтобы не отображать в автореферате
{...}

\newcommand{\defenseDate}              % Защита, дата
{«5» февраля 2025 года в 12 часов 30 минут}
\newcommand{\defenseDateEn}              % Защита, дата
{February 5, 2025, at 12:30 p.m.}
\newcommand{\defenseCouncilNumber}     % Защита, номер диссертационного совета
{1.2.2.2.}
\newcommand{\defenseCouncilNumberEn}     % Защита, номер диссертационного совета
{1.2.2.2.}
\newcommand{\defenseCouncilTitle}      % Защита, учреждение диссертационного совета
{\fixme{Название учреждения}}
\newcommand{\defenseCouncilTitleEn}      % Защита, учреждение диссертационного совета
{\fixme{Defence council title}}
\newcommand{\defenseCouncilAddress}    % Защита, адрес учреждение диссертационного совета
{Территория Инновационного Центра «Сколково», Большой бульвар д.30, стр.1, Москва 121205}
\newcommand{\defenseCouncilAddressEn}    % Защита, адрес учреждение диссертационного совета
{The territory of the Skolkovo Innovation Center, Bolshoy Boulevard, 30, p.1, Moscow 121205}
\newcommand{\defenseCouncilPhone}      % Телефон для справок
{\fixme{+7~(0000)~00-00-00}}

\newcommand{\defenseSecretaryFio}      % Секретарь диссертационного совета, ФИО
{Копелевич Григорий Александрович}
\newcommand{\defenseSecretaryRegalia}  % Секретарь диссертационного совета, регалии
{кандидат физико-математических наук}            % Для сокращений есть ГОСТы, например: ГОСТ Р 7.0.12-2011 + http://base.garant.ru/179724/#block_30000

\newcommand{\defenseSecretaryFioEn}      % Секретарь диссертационного совета, ФИО
{Kopelevich Grigoriy Aleksandrovich}
\newcommand{\defenseSecretaryRegaliaEn}  % Секретарь диссертационного совета, регалии
{Candidate of Physical and Mathematical Sciences}    

\newcommand{\synopsisLibrary}          % Автореферат, название библиотеки
{Сколтеха и на сайте организации https://dissovet.skoltech.ru}
\newcommand{\synopsisDate}             % Автореферат, дата рассылки
{<<\rule[-0.1cm]{0.75cm}{0.15mm}>>\rule[-0.1cm]{3cm}{0.15mm} \the\year~г}

\newcommand{\synopsisLibraryEn}          % Автореферат, название библиотеки
{library of Skoltech or on the website https://dissovet.skoltech.ru}
\newcommand{\synopsisDateEn}             % Автореферат, дата рассылки
{<<\rule[-0.1cm]{0.75cm}{0.15mm}>>\rule[-0.1cm]{3cm}{0.15mm}, \the\year}

% To avoid conflict with beamer class use \providecommand
\providecommand{\keywords}%            % Ключевые слова для метаданных PDF диссертации и автореферата
{}
