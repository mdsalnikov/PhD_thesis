\chapter{Conclusion and Future Work}
\label{chap:conclusion}

% 7.1 Synthesis of Contributions
\section{Synthesis of Contributions}
\label{sec:conclusion_synthesis}
% Comment: Provide a comprehensive summary of the thesis's achievements. Reiterate the 
% novel methods for KG-LLM fusion, the ShortPathQA benchmark contribution, system 
% demonstrations, and insights from related studies (LoRA). Emphasize the overall 
% narrative: identifying the LLM factuality problem and providing KG-based solutions.


% 7.2 Revisiting Research Questions
\section{Revisiting Research Questions}
\label{sec:conclusion_revisiting_rqs}
% Comment: Directly address the research questions posed in Chapter 1, providing 
% concise answers based on the evidence presented throughout the thesis.


% 7.3 Limitations of the Presented Work
\section{Limitations of the Presented Work}
\label{sec:conclusion_limitations}
% Comment: Honestly discuss the limitations of your research. This might include the 
% scope (e.g., focus on factoid QA), the specific KGs or LLMs used, scalability 
% challenges of the proposed methods, or aspects not fully explored.


% 7.4 Future Research Directions
\section{Future Research Directions}
\label{sec:conclusion_future_work}
% Comment: Propose concrete directions for future work building upon your thesis. 
% Examples: exploring more complex reasoning tasks, applying methods to different 
% domains or languages, integrating dynamic/temporal KGs, improving the efficiency 
% and scalability of fusion methods, developing more sophisticated control mechanisms, 
% investigating user interaction with KG-aware LLMs. 