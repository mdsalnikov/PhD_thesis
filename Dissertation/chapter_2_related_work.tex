\chapter{Background and Related Work}
\label{chap:related_work}

% 2.1 Large Language Models for Question Answering
\section{Large Language Models for Question Answering}
\label{sec:rw_llms_qa}
% Comment: Provide necessary background on LLMs (architectures like Transformers, 
% pre-training/fine-tuning paradigms) and their application to various QA formats 
% (e.g., extractive, abstractive, open-domain, closed-book). Discuss limitations 
% relevant to factuality.


% 2.2 Knowledge Graphs and Semantic Technologies
\section{Knowledge Graphs and Semantic Technologies}
\label{sec:rw_kgs}
% Comment: Explain foundational concepts of KGs (nodes, edges, triples), 
% representation standards (RDF, OWL), query languages (SPARQL), and prominent public 
% KGs (Wikidata, DBpedia).


% 2.3 Knowledge Graph Question Answering (KGQA)
\section{Knowledge Graph Question Answering (KGQA)}
\label{sec:rw_kgqa}
% Comment: Review traditional KGQA methods that predate heavy LLM integration (e.g., 
% semantic parsing to SPARQL, embedding-based methods). You could briefly mention 
% "Konstruktor" here as an example of a simple baseline KGQA system, positioning it 
% relative to the more complex LLM-integrated approaches you focus on later.


% 2.4 Integrating Large Language Models and Knowledge Graphs
\section{Integrating Large Language Models and Knowledge Graphs}
\label{sec:rw_llm_kg_integration}
% Comment: This is a key section. Survey the existing literature on combining LLMs and 
% KGs. Categorize different approaches (e.g., KG-augmented retrieval, using LLMs to 
% generate KG queries, using KGs to verify/rerank LLM outputs, joint modeling). 
% Position your work within this landscape, highlighting how your methods (especially 
% the controllable fusion approach) differ from or improve upon prior work. Mention 
% the "How Much Knowledge Can You Pack..." paper here to contrast external KG 
% integration with internal knowledge packing via adapters like LoRA.


% 2.5 Related QA Tasks: Multilingual and Comparative QA
\section{Related QA Tasks: Multilingual and Comparative QA}
\label{sec:rw_related_qa}
% Comment: Briefly discuss related work specifically in multilingual QA (context for "A 
% System for Answering Simple Questions...") and comparative QA (context for "CAM 
% 2.0..."). Explain how insights or techniques from these areas relate to your core 
% focus on factoid QA with KGs. 