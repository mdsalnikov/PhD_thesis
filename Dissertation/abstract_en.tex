\chapter*{Abstract}

The most flexible and general approach to modeling random perturbations in constrained optimization is the use of Chance Constraints (CC). This type of constraint allows for the pre-setting of the probability of violating the initial constraints, thereby preventing excessive conservatism in the solution. Generally, CCs do not admit a closed-form expression, which prevents their direct use in numerical methods. To circumvent this limitation, various types of data-driven approximations have been proposed, notably the Scenario Approximation (SA). Despite theoretical guarantees for obtaining a feasibe solution with high probability, the required amount of data (scenarios) is high, leading to a high-dimensional optimization problem. This dissertation proposes methods and algorithms for evaluating the CC value and solving optimization problems with CCs, which are significantly less demanding in terms of the amount of data (scenarios) needed to obtain a feasible solution with high probability.

A method for evaluating the value of CC was developed using an optimization-statistical approach of adaptive importance sampling and demonstrated for the feasibility estimation of the current generation mode in the power grid. Two approaches to scenario selection for SA were proposed for the optimization with additive and multiplicative uncertainty. The basis of the approach is the identification of the region of redundant scenarios that are not representative in terms of constraint violations. Their efficiency is demonstrated for optimal power flow problem.

According to the research results, a significant improvement in the convergence rate of the variance of the estimate to the minimum was demonstrated for evaluating the value of CC, reducing dependence on the dimensionality of the problem to $O(\sqrt{\log K})$, where 
$K$ is the number of constraints. The previous results claim linear dependence  - $O(K)$. Numerical experiments showed that the proposed method is more stable in certain synthetic settings, compared to other modern methods and more efficient for electrical networks setups. For optimization with CC, it was theoretically shown that the number of required scenarios to obtain a feasible solution with high probability is reduced, and numerically, the required number of scenarios was reduced by an average of 2 times.