\chapter{Application: Enhancing Comparative Question Answering with LLMs}
\label{ch:comparative_qa}

\section{Introduction}
\label{sec:compqa:intro}
% TODO: Introduce Comparative QA and the CAM 2.0 system.

\section{The CAM 2.0 System Architecture}
\label{sec:compqa:cam2_arch}
% TODO: Describe the overall architecture of CAM 2.0.

\section{LLM Integration and Fine-tuning}
\label{sec:compqa:llm_integration}
% TODO: Detail your contributions: LLM selection (Llama, Vicuna), fine-tuning process, integration challenges.

\section{Experimental Setup}
\label{sec:compqa:setup}
% TODO: Describe datasets, baselines, and evaluation metrics used for CAM 2.0.

\section{Results and Analysis}
\label{sec:compqa:results}
% TODO: Present the performance results, focusing on the impact of the LLM components you implemented.

\section{Discussion}
\label{sec:compqa:discussion}
% TODO: Discuss the effectiveness of LLMs in this complex QA task. Potential connections to KG methods.

\section{Chapter Summary}
\label{sec:compqa:summary}
% TODO: Summarize the findings regarding LLMs in comparative QA. 