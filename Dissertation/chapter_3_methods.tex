\chapter{Methods for Fusing LLMs and KGs for Enhanced Factuality}
\label{chap:methods}

% Comment: This chapter details your core methodological contributions, drawing heavily 
% from "Answer Candidate Type Selection...", "Large Language Models Meet KGs...", 
% and the methodological aspects of "ShortPathQA...

% 3.1 Guiding LLM Generation with KG Constraints: Answer Type Selection
\section{Guiding LLM Generation with KG Constraints: Answer Type Selection}
\label{sec:methods_type_selection}
% Comment: Describe the approach from "Answer Candidate Type Selection...". Explain 
% the motivation (reducing hallucination by constraining output types) and the 
% mechanism (using KG information to predict/select answer types for the LLM). 
% Detail the model architecture and process. This represents an 
% early/foundational method you developed.
% You can \chapter{Core Method III: KG-Guided Answer Candidate Type Selection}
\label{ch:type_selection}

\section{Introduction}
\label{sec:typesel:intro}
% TODO: Motivate the need for answer type selection in QA.

\section{Methodology}
\label{sec:typesel:method}
% TODO: Describe the approach combining Text-to-Text LMs and KGs.
% TODO: Explain how KGs inform the type selection process.

\section{Experimental Setup}
\label{sec:typesel:setup}
% TODO: Describe datasets, models, and evaluation metrics used.

\section{Results and Analysis}
\label{sec:typesel:results}
% TODO: Present the experimental results demonstrating the effectiveness of the method.

\section{Discussion}
\label{sec:typesel:discussion}
% TODO: Discuss the benefits and limitations. How does this complement other methods?

\section{Chapter Summary}
\label{sec:typesel:summary}
% TODO: Summarize the main contributions and findings of this chapter.  here.


% 3.2 Controllable Fusion using Knowledge Graph Paths
\section{Controllable Fusion using Knowledge Graph Paths}
\label{sec:methods_kg_path_fusion}
% Comment: This section details the core method from "Large Language Models Meet KGs..." 
% and refined/formalized in "ShortPathQA...". Explain the concept of using 
% explicit KG paths (specifically shortest paths) to provide factual evidence or 
% structural guidance to the LLM during the QA process. Describe the architecture, 
% the fusion mechanism (how path information is integrated with the LLM), and how 
% it enables more *controllable* generation compared to less constrained methods.
% You can \chapter{Core Method I: Knowledge Graph-based Reranking of LLM Answers}
\label{ch:reranking}

\section{Introduction}
\label{sec:rerank:intro}
% TODO: Introduce the problem of LLM hallucination and the need for reranking.
% TODO: Frame this as building upon the foundational work.

\section{Methodology: KG-based Reranking}
\label{sec:rerank:method}
% TODO: Detail the reranking approach from the SWJ paper.
% TODO: Emphasize your key ideas and contributions.

\section{Subgraph Generation for Reranking}
\label{sec:rerank:subgraphs}
% TODO: Describe the subgraph generation techniques used specifically for reranking.
% TODO: Highlight your contributions to the code and ideas.

\section{Experimental Setup}
\label{sec:rerank:setup}
% TODO: Describe datasets, baselines (including the initial LLM+KG approach), metrics.

\section{Results and Analysis}
\label{sec:rerank:results}
% TODO: Present the results demonstrating the effectiveness of the reranking method.
% TODO: Detail the specific experiments you conducted.

\section{Discussion}
\label{sec:rerank:discussion}
% TODO: Analyze the strengths, weaknesses, and improvements over the foundational approach.

\section{Chapter Summary}
\label{sec:rerank:summary}
% TODO: Summarize the main contributions related to KG-based reranking.  here.


% 3.3 Experimental Design and Baselines
\section{Experimental Design and Baselines}
\label{sec:methods_experimental_design}
% Comment: Describe the experimental setup used to evaluate the methods in 3.1 and 3.2. 
% Specify the datasets used (you might briefly introduce the motivation for ShortPathQA 
% here, leading into the next chapter), the evaluation metrics focused on factuality 
% and accuracy, and the baseline models (including perhaps zero-shot/few-shot LLMs, 
% and simpler KGQA systems).


% 3.4 Results and Analysis: Demonstrating Factuality Improvements
\section{Results and Analysis: Demonstrating Factuality Improvements}
\label{sec:methods_results}
% Comment: Present the quantitative results comparing your methods against baselines. 
% Crucially, analyze *why* your methods work, using examples and error analysis to 
% show how KG fusion mitigates specific failure modes of standalone LLMs (e.g., 
% hallucinations, factual errors). Discuss the trade-offs of each method. 