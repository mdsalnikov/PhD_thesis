\chapter*{Аннотация}

Наиболее гибкий и общий подход к моделированию случайных возмущений в условных задачах оптимизации — использование \emph{Вероятностных Ограничений} (ВО). Они позволяют наперед задавать вероятность нарушения исходных ограничений и избегать излишней консервативности. В большинстве случаев, ВО не выражаются через элементарные функции, что затрудняет их использование в численных методах. Чтобы обойти это, были предложены различные аппроксимации с использованием данных, включая Аппроксимацию Сценариями (АС). Несмотря на теоретические гарантии, необходимое количество данных (сценариев) велико, что усложняет оптимизацию. В данной работе предлагаются методы и алгоритмы для оценки значения ВО и решения задач оптимизации с ВО, требующие меньше данных для получения приближенного решения, допустимого для ВО с высокой вероятностью.

Метод для оценки значения ВО  разработан с использованием оптимизационно-статистического подхода адаптивной сыборки по значимости и продемонстрирован на примере оценки допустимости текущего режима генерации в элетрических сетях. Предложены подходы к подбору сценариев для АС для линейного программирования в случае аддитивных и мультипликативных возмущений, выделяющий область избыточных сценариев, не приводящих к выходу за допустимую область. Эффективность подходов продемонстрирована на примере задачи оптимального распределения потоков электроэнергии.

Результаты исследования показали значительное улучшение скорости сходимости дисперсии оценки к минимуму и снижение зависимости от размерности задачи до $O(\sqrt{\log K})$, где $K$ — количество детерменированных ограничений. Предыдущие результаты в области показывали линейную зависимость - $O(K)$. Численные эксперименты показали, что предложенный метод более устойчив в определенных синтетических постановках по сравнению с другими современными методами и более эффективен в приложениях из энергетических сетей. Для оптимизации с ВО удалось теоретически доказать уменьшение количества необходимых сценариев для получения допустимого решения с высокой вероятностью; численно, количество сценариев сократилось, в среднем, в 2 раза.