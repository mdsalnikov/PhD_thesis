\chapter*{Аннотация}
Сочетание больших языковых моделей и графов знаний для ответов на вопросы нацелено на использование языковых навыков моделей и фактической точности графов. Однако языковые модели часто создают «галлюцинации». Данная диссертация решает эти проблемы, предлагая и тестируя новые методы для надежного объединения языковых моделей и графов знаний с целью улучшения точности, надежности и управляемости вопросно-ответных систем.

Сделаны два основных вклада. Первый — это отбор кандидатов в ответы по типу. Этот метод улучшает генерацию ответов, используя способность языковой модели предсказывать семантический тип ответа, даже если первоначальный фактический ответ неверен. Эта информация о типе, вместе с правилами типов из графа знаний, помогает фильтровать и переранжировать кандидатов. Эксперименты на таких наборах данных, как SimpleQuestions-Wikidata, RuBQ и Mintaka, показывают стабильный прирост Hits@1 для различных языковых моделей, превосходя стандартные модели.

Второй ключевой вклад — это система для контролируемого объединения с использованием переранжирования на основе подграфов. Этот метод повышает фактическую корректность ответов, сгенерированных языковыми моделями, проверяя несколько кандидатов по данным из подграфов графа знаний. Эти подграфы связывают сущности вопроса с кандидатами. Процесс включает извлечение подграфов, создание признаков из них и использование различных моделей ранжирования. Это последовательно улучшает выбор ответа. Для содействия исследованиям в этой области был создан набор данных ShortPathQA, предлагающий вопросы с готовыми подграфами графа знаний.

Практические системы с API-интерфейсами и инструментом визуализации подграфов показывают, что эти методы могут использоваться в реальных приложениях. Они служат примерами и инструментами для исследователей.

В итоге, данная диссертация дает новое понимание совместной работы языковых моделей и графов знаний. Она предлагает методы для создания более точных, надежных и понятных вопросно-ответных систем для графов знаний. Эти техники улучшают фактологическую основу и поддерживают создание более контролируемого искусственного интеллекта.