\chapter{Core Method I: Knowledge Graph-based Reranking of LLM Answers}
\label{ch:reranking}

\section{Introduction}
\label{sec:rerank:intro}
% TODO: Introduce the problem of LLM hallucination and the need for reranking.
% TODO: Frame this as building upon the foundational work.

\section{Methodology: KG-based Reranking}
\label{sec:rerank:method}
% TODO: Detail the reranking approach from the SWJ paper.
% TODO: Emphasize your key ideas and contributions.

\section{Subgraph Generation for Reranking}
\label{sec:rerank:subgraphs}
% TODO: Describe the subgraph generation techniques used specifically for reranking.
% TODO: Highlight your contributions to the code and ideas.

\section{Experimental Setup}
\label{sec:rerank:setup}
% TODO: Describe datasets, baselines (including the initial LLM+KG approach), metrics.

\section{Results and Analysis}
\label{sec:rerank:results}
% TODO: Present the results demonstrating the effectiveness of the reranking method.
% TODO: Detail the specific experiments you conducted.

\section{Discussion}
\label{sec:rerank:discussion}
% TODO: Analyze the strengths, weaknesses, and improvements over the foundational approach.

\section{Chapter Summary}
\label{sec:rerank:summary}
% TODO: Summarize the main contributions related to KG-based reranking. 