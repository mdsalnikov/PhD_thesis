\chapter{ShortPathQA: A Benchmark for Knowledge Graph Question Answering}
\label{chap:shortpathqa}

\section{Introduction}
\label{sec:shortpath:intro}
Building on the methods for answer candidate generation (Chapter~\ref{chap:candidate_generation}) and controllable fusion using knowledge graph paths (Chapter~\ref{chap:controllable_fusion}), this chapter introduces ShortPathQA, a novel benchmark for directly evaluating knowledge graph question answering capabilities. While the previous chapters focused on techniques to enhance LLM-generated answers with knowledge graphs, ShortPathQA explores a complementary approach that focuses on extracting precise answers directly from knowledge graph paths.

\section{The ShortPathQA Approach}
\label{sec:shortpath:method}
% TODO: Describe the methodology from the NLDB paper.

\section{Dataset Creation}
\label{sec:shortpath:dataset}
% TODO: Detail your work on creating the ShortPathQA dataset.

\section{Experimental Setup}
\label{sec:shortpath:setup}
% TODO: Describe the experiments conducted (baselines, metrics).

\section{Results and Analysis}
\label{sec:shortpath:results}
% TODO: Present the full experimental evaluation results.
% TODO: Compare performance characteristics with other potential approaches.

\section{Discussion}
\label{sec:shortpath:discussion}
% TODO: Discuss the strengths (e.g., precision) and weaknesses (e.g., coverage, flexibility) of this method.

\section{Chapter Summary}
\label{sec:shortpath:summary}
% TODO: Summarize the main contributions and findings regarding ShortPathQA. 