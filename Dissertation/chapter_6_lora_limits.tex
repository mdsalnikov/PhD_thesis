\chapter{On the Limits of Internalizing Knowledge: Insights from LoRA}
\label{chap:lora_limits}

% Comment: This chapter incorporates the findings from "How Much Knowledge Can You 
% Pack...", positioning it as an exploration complementary to the main KG fusion 
% theme.

% 6.1 Knowledge Integration Strategies: External vs. Internal
\section{Knowledge Integration Strategies: External vs. Internal}
\label{sec:lora_external_vs_internal}
% Comment: Revisit the idea of different ways to provide knowledge to LLMs. Contrast 
% the *external* approach (using KGs, as explored in Chapters 3-5) with *internal* 
% approaches like fine-tuning or parameter modification using methods like LoRA.


% 6.2 Investigating Knowledge Capacity in LoRA Adapters
\section{Investigating Knowledge Capacity in LoRA Adapters}
\label{sec:lora_capacity_investigation}
% Comment: Describe the study from "How Much Knowledge Can You Pack...". Explain the 
% methodology for "packing" knowledge into LoRA adapters and for measuring the 
% trade-off between acquired knowledge and the preservation of the base LLM's general 
% abilities.


% 6.3 Findings and Implications for KG-LLM Approaches
\section{Findings and Implications for KG-LLM Approaches}
\label{sec:lora_implications}
% Comment: Present the key results regarding the capacity limits of LoRA for storing 
% knowledge. Discuss the implications: Do these findings suggest inherent limitations 
% to solely relying on internalizing knowledge within LLM parameters, especially for 
% vast, dynamic factual information? Argue how these results might strengthen the case 
% for hybrid approaches that leverage external, structured knowledge sources like KGs 
% for robust factuality. 