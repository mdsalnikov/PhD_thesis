%&preformat-disser
\RequirePackage[l2tabu,orthodox]{nag} % Раскомментировав, можно в логе получать рекомендации относительно правильного использования пакетов и предупреждения об устаревших и нерекомендуемых пакетах
% Формат А4, 14pt (ГОСТ Р 7.0.11-2011, 5.3.6)
\documentclass[a4paper,14pt,oneside,openany]{memoir}

\input{common/setup}            % общие настройки шаблона
\input{common/packages}         % Пакеты общие для диссертации и автореферата
\synopsisfalse                      % Этот документ --- не автореферат
\input{Dissertation/dispackages}    % Пакеты для диссертации
\input{Dissertation/userpackages}   % Пакеты для специфических пользовательских задач

\input{Dissertation/setup}      % Упрощённые настройки шаблона

% Новые переменные, которые могут использоваться во всём проекте
% ГОСТ 7.0.11-2011
% 9.2 Оформление текста автореферата диссертации
% 9.2.1 Общая характеристика работы включает в себя следующие основные структурные
% элементы:
% актуальность темы исследования;
\newcommand{\actualityTXT}{Актуальность темы.}
% степень ее разработанности;
\newcommand{\progressTXT}{Степень разработанности темы.}
% цели и задачи;
\newcommand{\aimTXT}{Целью}
\newcommand{\tasksTXT}{задачи}
% научную новизну;
\newcommand{\noveltyTXT}{Научная новизна:}
% теоретическую и практическую значимость работы;
%\newcommand{\influenceTXT}{Теоретическая и практическая значимость}
% или чаще используют просто
\newcommand{\influenceTXT}{Практическая значимость}
% методологию и методы исследования;
\newcommand{\methodsTXT}{Методология и методы исследования.}
% положения, выносимые на защиту;
\newcommand{\defpositionsTXT}{Основные положения, выносимые на~защиту:}
% степень достоверности и апробацию результатов.
\newcommand{\reliabilityTXT}{Достоверность}
\newcommand{\probationTXT}{Апробация работы.}

\newcommand{\contributionTXT}{Личный вклад.}
\newcommand{\publicationsTXT}{Публикации.}


%%% Заголовки библиографии:

% для автореферата:
\newcommand{\bibtitleauthor}{Публикации автора по теме диссертации}
\newcommand{\bibtitleauthorEn}{Author's publications on the dissertation subject}


% для стиля библиографии `\insertbiblioauthorgrouped`
\newcommand{\bibtitleauthorvak}{В изданиях из списка ВАК РФ}
\newcommand{\bibtitleauthorscopus}{В изданиях, входящих в международную базу цитирования Scopus}
\newcommand{\bibtitleauthorwos}{В изданиях, входящих в международную базу цитирования Web of Science}
\newcommand{\bibtitleauthorother}{В прочих изданиях}
\newcommand{\bibtitleauthorconf}{В сборниках трудов конференций}
\newcommand{\bibtitleauthorpatent}{Зарегистрированные патенты}
\newcommand{\bibtitleauthorprogram}{Зарегистрированные программы для ЭВМ}

% для стиля библиографии `\insertbiblioauthorimportant`:
\newcommand{\bibtitleauthorimportant}{Наиболее значимые \protect\MakeLowercase\bibtitleauthor}

% для списка литературы в диссертации и списка чужих работ в автореферате:
\newcommand{\bibtitlefull}{Список литературы} % (ГОСТ Р 7.0.11-2011, 4)
\newcommand{\bibtitlefullEn}{Bibliography}



% Aliases for popular symbols
\newtheorem{lemma}{Lemma}
% \newtheorem{theorem}{Theorem}
% \newtheorem{remark}{Remark}
\newtheorem{assumption}{Assumption}
\newtheorem{corollary}{Corollary}
\DeclareMathOperator{\cost}{cost}
\DeclareMathOperator{\kl}{KL}


\newcommand{\bp}{\mathbf{p}}
\newcommand{\bx}{\mathbf{x}}
\newcommand{\bV}{\mathbb{V}}
\newcommand{\bom}{\boldsymbol\omega}
\newcommand{\bmu}{\boldsymbol\mu}
\newcommand{\bth}{\boldsymbol\theta}
\newcommand{\cE}{{\cal E}}
\newcommand{\cV}{{\cal V}}
%\newcommand{\cP}{{\cal P}}
\newcommand{\cN}{{\cal N}}
\newcommand{\cG}{{\cal G}}

\DeclareMathOperator*{\argmax}{argmax}
\DeclareMathOperator*{\argmin}{argmin}
\newcommand{\mc}[1]{\mathcal{#1}}
\newcommand{\id}{\mathbbm{1}}
\newcommand{\rmi}{\mathrm{i}}
\newcommand{\placeholder}{\ast}

\DeclareMathOperator{\Var}{Var}

\tikzset{blank/.style={rectangle,inner sep=0pt,draw=none,fill=none,minimum
    size=0pt} }         % Новые переменные, для всего проекта

%%% Основные сведения %%%
\newcommand{\thesisAuthorLastName}{Лукашевич}
\newcommand{\thesisAuthorOtherNames}{Александр Леонидович}
\newcommand{\thesisAuthorInitials}{А.\,Л.}
\newcommand{\thesisAuthorLastNameEn}{Lukashevich}
\newcommand{\thesisAuthorOtherNamesEn}{Aleksandr Leonidovich}
\newcommand{\thesisAuthorInitialsEn}{A.\,L.}

\newcommand{\thesisAuthor}             % Диссертация, ФИО автора
{%
    \texorpdfstring{% \texorpdfstring takes two arguments and uses the first for (La)TeX and the second for pdf
        \thesisAuthorLastName~\thesisAuthorOtherNames% так будет отображаться на титульном листе или в тексте, где будет использоваться переменная
    }{%
        \thesisAuthorLastName, \thesisAuthorOtherNames% эта запись для свойств pdf-файла. В таком виде, если pdf будет обработан программами для сбора библиографических сведений, будет правильно представлена фамилия.
    }
}
\newcommand{\thesisAuthorEn}             % Диссертация, ФИО автора
{%
    \texorpdfstring{% \texorpdfstring takes two arguments and uses the first for (La)TeX and the second for pdf
        \thesisAuthorLastNameEn~\thesisAuthorOtherNamesEn% так будет отображаться на титульном листе или в тексте, где будет использоваться переменная
    }{%
        \thesisAuthorLastNameEn, \thesisAuthorOtherNamesEn% эта запись для свойств pdf-файла. В таком виде, если pdf будет обработан программами для сбора библиографических сведений, будет правильно представлена фамилия.
    }
}

\newcommand{\thesisAuthorShort}        % Диссертация, ФИО автора инициалами
{\thesisAuthorInitials~\thesisAuthorLastName}
\newcommand{\thesisAuthorShortEn}        % Диссертация, ФИО автора инициалами
{\thesisAuthorInitialsEn~\thesisAuthorLastNameEn}

%\newcommand{\thesisUdk}                % Диссертация, УДК
%{\fixme{xxx.xxx}}
\newcommand{\thesisTitle}              % Диссертация, название
{Эффективные подходы на основе данных в задачах стохастического оптимального распределения потоков электроэнергии}
\newcommand{\thesisTitleEn}              % Диссертация, название
{Efficient Data-Driven Approaches in Stochastic Optimal Power Flow}
\newcommand{\thesisSpecialtyNumber}    % Диссертация, специальность, номер
{1.2.2}%05.13.18
\newcommand{\thesisSpecialtyTitle}     % Диссертация, специальность, название (название взято с сайта ВАК для примера)
{Математическое моделирование, численные методы и комплексы программ}
\newcommand{\thesisSpecialtyTitleEn}     % Диссертация, специальность, название (название взято с сайта ВАК для примера)
{Mathematical modeling, numerical methods and program complexes}
%% \newcommand{\thesisSpecialtyTwoNumber} % Диссертация, вторая специальность, номер
%% {\fixme{XX.XX.XX}}
%% \newcommand{\thesisSpecialtyTwoTitle}  % Диссертация, вторая специальность, название
%% {\fixme{Теория и~методика физического воспитания, спортивной тренировки,
%% оздоровительной и~адаптивной физической культуры}}
\newcommand{\thesisDegree}             % Диссертация, ученая степень
{кандидата физико-математических наук}
\newcommand{\thesisDegreeEn}             % Диссертация, ученая степень
{Candidate of Physical and Mathematical Sciences}
\newcommand{\thesisDegreeShort}        % Диссертация, ученая степень, краткая запись
{канд.~физ.-мат.~наук}
\newcommand{\thesisDegreeShortEn}        % Диссертация, ученая степень, краткая запись
{cand.~of phys.-math.~sciences}
\newcommand{\thesisCity}               % Диссертация, город написания диссертации
{Москва}
\newcommand{\thesisCityEn}               % Диссертация, город написания диссертации
{Moscow}
\newcommand{\thesisYear}               % Диссертация, год написания диссертации
{2024}
\newcommand{\thesisOrganization}       % Диссертация, организация
{Автономная некоммерческая образовательная организация высшего профессионального образования <<Сколковский институт науки и технологий>>}
\newcommand{\thesisOrganizationEn}       % Диссертация, организация
{Autonomous Non-Profit Organization for Higher Education <<Skolkovo Institute of Science and Technology>>}
\newcommand{\thesisOrganizationShort}  % Диссертация, краткое название организации для доклада
{Сколтех}
\newcommand{\thesisOrganizationShortEn}  % Диссертация, краткое название организации для доклада
{Skoltech}

\newcommand{\thesisInOrganization}     % Диссертация, организация в предложном падеже: Работа выполнена в ...
{Автономной некоммерческой образовательной организации высшего профессионального образования <<Сколковский институт науки и технологий>>}

%% \newcommand{\supervisorDead}{}           % Рисовать рамку вокруг фамилии
\newcommand{\supervisorFio}              % Научный руководитель, ФИО
{Грязина Елена Николаевна }
\newcommand{\supervisorRegalia}          % Научный руководитель, регалии
{доктор компьютерных наук, профессор}
\newcommand{\supervisorFioShort}         % Научный руководитель, ФИО
{Е.\,Н.~Грязина}
\newcommand{\supervisorRegaliaShort}     % Научный руководитель, регалии
{д.~к.~н, проф.}

\newcommand{\supervisorFioEn}              % Научный руководитель, ФИО
{Gryazina Elena Nikolaevna }
\newcommand{\supervisorRegaliaEn}          % Научный руководитель, регалии
{Doctor of Computer Sciences, Professor}
\newcommand{\supervisorFioShortEn}         % Научный руководитель, ФИО
{\fixme{E.\,N.~Gryazina}}
\newcommand{\supervisorRegaliaShortEn}     % Научный руководитель, регалии
{\fixme{D.Sc., Prof}}

%% \newcommand{\supervisorTwoDead}{}        % Рисовать рамку вокруг фамилии
%% \newcommand{\supervisorTwoFio}           % Второй научный руководитель, ФИО
%% {\fixme{Фамилия Имя Отчество}}
%% \newcommand{\supervisorTwoRegalia}       % Второй научный руководитель, регалии
%% {\fixme{уч. степень, уч. звание}}
%% \newcommand{\supervisorTwoFioShort}      % Второй научный руководитель, ФИО
%% {\fixme{И.\,О.~Фамилия}}
%% \newcommand{\supervisorTwoRegaliaShort}  % Второй научный руководитель, регалии
%% {\fixme{уч.~ст.,~уч.~зв.}}

\newcommand{\opponentOneFio}           % Оппонент 1, ФИО
{\fixme{Фамилия Имя Отчество}}
\newcommand{\opponentOneRegalia}       % Оппонент 1, регалии
{\fixme{доктор физико-математических наук, профессор}}
\newcommand{\opponentOneJobPlace}      % Оппонент 1, место работы
{\fixme{Не очень длинное название для места работы}}
\newcommand{\opponentOneJobPost}       % Оппонент 1, должность
{\fixme{старший научный сотрудник}}

\newcommand{\opponentTwoFio}           % Оппонент 2, ФИО
{\fixme{Фамилия Имя Отчество}}
\newcommand{\opponentTwoRegalia}       % Оппонент 2, регалии
{\fixme{кандидат физико-математических наук}}
\newcommand{\opponentTwoJobPlace}      % Оппонент 2, место работы
{\fixme{Основное место работы c длинным длинным длинным длинным названием}}
\newcommand{\opponentTwoJobPost}       % Оппонент 2, должность
{\fixme{старший научный сотрудник}}

%% \newcommand{\opponentThreeFio}         % Оппонент 3, ФИО
%% {\fixme{Фамилия Имя Отчество}}
%% \newcommand{\opponentThreeRegalia}     % Оппонент 3, регалии
%% {\fixme{кандидат физико-математических наук}}
%% \newcommand{\opponentThreeJobPlace}    % Оппонент 3, место работы
%% {\fixme{Основное место работы c длинным длинным длинным длинным названием}}
%% \newcommand{\opponentThreeJobPost}     % Оппонент 3, должность
%% {\fixme{старший научный сотрудник}}

\newcommand{\opponentOneFioEn}           % Оппонент 1, ФИО
{\fixme{Name name name}}
\newcommand{\opponentOneRegaliaEn}       % Оппонент 1, регалии
{\fixme{Doctor of physcial and mathematical sciences, Professor}}
\newcommand{\opponentOneJobPlaceEn}      % Оппонент 1, место работы
{\fixme{Somewhat long job place name}}
\newcommand{\opponentOneJobPostEn}       % Оппонент 1, должность
{\fixme{Senior research scientist}}

\newcommand{\opponentTwoFioEn}           % Оппонент 2, ФИО
{\fixme{Name name name}}
\newcommand{\opponentTwoRegaliaEn}       % Оппонент 2, регалии
{\fixme{Doctor of physcial and mathematical sciences}}
\newcommand{\opponentTwoJobPlaceEn}      % Оппонент 2, место работы
{\fixme{Main job place with a long long long long long long, reeeeally long title}}
\newcommand{\opponentTwoJobPostEn}       % Оппонент 2, должность
{\fixme{Senior research scientist}}

\newcommand{\leadingOrganizationTitle} % Ведущая организация, дополнительные строки. Удалить, чтобы не отображать в автореферате
{...}
\newcommand{\leadingOrganizationTitleEn} % Ведущая организация, дополнительные строки. Удалить, чтобы не отображать в автореферате
{...}

\newcommand{\defenseDate}              % Защита, дата
{11 сентября 2024г.~в 16 часов 00 минут}
\newcommand{\defenseDateEn}              % Защита, дата
{11 September 2024 at 16:00}
\newcommand{\defenseCouncilNumber}     % Защита, номер диссертационного совета
{...}
\newcommand{\defenseCouncilNumberEn}     % Защита, номер диссертационного совета
{\fixme{D\,123.456.78}}
\newcommand{\defenseCouncilTitle}      % Защита, учреждение диссертационного совета
{\fixme{Название учреждения}}
\newcommand{\defenseCouncilTitleEn}      % Защита, учреждение диссертационного совета
{\fixme{Defence council title}}
\newcommand{\defenseCouncilAddress}    % Защита, адрес учреждение диссертационного совета
{Территория Инновационного Центра «Сколково», Большой бульвар д.30, стр.1, Москва 121205}
\newcommand{\defenseCouncilAddressEn}    % Защита, адрес учреждение диссертационного совета
{The territory of the Skolkovo Innovation Center, Bolshoy Boulevard, 30, p.1, Moscow 121205}
\newcommand{\defenseCouncilPhone}      % Телефон для справок
{\fixme{+7~(0000)~00-00-00}}

\newcommand{\defenseSecretaryFio}      % Секретарь диссертационного совета, ФИО
{Копелевич Григорий Александрович}
\newcommand{\defenseSecretaryRegalia}  % Секретарь диссертационного совета, регалии
{\fixme{...}}            % Для сокращений есть ГОСТы, например: ГОСТ Р 7.0.12-2011 + http://base.garant.ru/179724/#block_30000

\newcommand{\defenseSecretaryFioEn}      % Секретарь диссертационного совета, ФИО
{Kopelevich Grigoriy Aleksandrovich}
\newcommand{\defenseSecretaryRegaliaEn}  % Секретарь диссертационного совета, регалии
{\fixme{...}}    

\newcommand{\synopsisLibrary}          % Автореферат, название библиотеки
{Сколтеха и на сайте организации https://dissovet.skoltech.ru}
\newcommand{\synopsisDate}             % Автореферат, дата рассылки
{<<\rule[-0.1cm]{0.75cm}{0.15mm}>>\rule[-0.1cm]{3cm}{0.15mm} \the\year~г}

\newcommand{\synopsisLibraryEn}          % Автореферат, название библиотеки
{library of Skoltech or on the website https://dissovet.skoltech.ru}
\newcommand{\synopsisDateEn}             % Автореферат, дата рассылки
{<<\rule[-0.1cm]{0.75cm}{0.15mm}>>\rule[-0.1cm]{3cm}{0.15mm}, \the\year}

% To avoid conflict with beamer class use \providecommand
\providecommand{\keywords}%            % Ключевые слова для метаданных PDF диссертации и автореферата
{}
             % Основные сведения
\input{common/fonts}            % Определение шрифтов (частичное)
\ifnumequal{\value{englishthesis}}{1}{
    \input{common/styles_en}           % Стили общие для диссертации и автореферата
}{
    \input{common/styles}           % Стили общие для диссертации и автореферата
}

\input{Dissertation/disstyles}  % Стили для диссертации
\newcommand\blank[1][\textwidth]{\noindent\rule[-.2ex]{#1}{.4pt}}

\theoremstyle{definition}
\newtheorem{definition}{Definition}

\theoremstyle{definition}
\newtheorem{example}{Example}

\theoremstyle{plain}
\newtheorem{theorem}{Theorem}

\newtheorem{proposition}{Proposition}

\theoremstyle{remark}
\newtheorem*{remark}{Remark} % Стили для специфических пользовательских задач

%%% Библиография. Выбор движка для реализации %%%
% Здесь только проверка установленного ключа. Сама настройка выбора движка
% размещена в common/setup.tex
\ifnumequal{\value{bibliosel}}{0}{%
    \input{biblio/predefined}   % Встроенная реализация с загрузкой файла через движок bibtex8
}{
    \input{biblio/biblatex}     % Реализация пакетом biblatex через движок biber
}

% Вывести информацию о выбранных опциях в лог сборки
\typeout{Selected options:}
\typeout{Draft mode: \arabic{draft}}
\typeout{Font: \arabic{fontfamily}}
\typeout{AltFont: \arabic{usealtfont}}
\typeout{Bibliography backend: \arabic{bibliosel}}
\typeout{Precompile images: \arabic{imgprecompile}}
% Вывести информацию о версиях используемых библиотек в лог сборки
\listfiles

%%% Управление компиляцией отдельных частей диссертации %%%
% Необходимо сначала иметь полностью скомпилированный документ, чтобы все
% промежуточные файлы были в наличии
% Затем, для вывода отдельных частей можно воспользоваться командой \includeonly
% Ниже примеры использования команды:
%
%\includeonly{Dissertation/part2}
%\includeonly{Dissertation/contents,Dissertation/appendix,Dissertation/conclusion}
%
% Если все команды закомментированы, то документ будет выведен в PDF файл полностью

\begin{document}
%%% Переопределение именований типовых разделов
\ifnumequal{\value{englishthesis}}{0}{
    % https://tex.stackexchange.com/a/156050
    \gappto\captionsrussian{\input{common/renames}\unskip} % for polyglossia and babel
    \input{common/renames}
}{}



%%% Структура диссертации (ГОСТ Р 7.0.11-2011, 4)
\thispagestyle{empty}

\noindent%
\begin{tabularx}{\textwidth}{@{}lXr@{}}%
    & & \textit{\textbf{На правах рукописи}}\\
    % \IfFileExists{images/logo.pdf}{\includegraphics[height=0.5cm]{logo}}{\rule[0pt]{0pt}{2.5cm}}  & &
    \ifnumequal{\value{showperssign}}{0}{%
        \rule[0pt]{0pt}{1.5cm}
    }{
        \includegraphics[height=1.5cm]{personal-signature.png}
    }\\
\end{tabularx}

\vspace{0pt plus9fill} %число перед fill = кратность относительно некоторого расстояния fill, кусками которого заполнены пустые места
\begin{center}
\textbf {\large \thesisAuthor}
% \end{center}

\vspace{0pt plus1fill} %число перед fill = кратность относительно некоторого расстояния fill, кусками которого заполнены пустые места
% \begin{center}
\textbf {\large %\MakeUppercase
\thesisTitle}

\vspace{0pt plus1fill} %число перед fill = кратность относительно некоторого расстояния fill, кусками которого заполнены пустые места
{\large \textbf{Специальность: \thesisSpecialtyNumber.~\thesisSpecialtyTitle}}

\ifdefined\thesisSpecialtyTwoNumber
{\large Специальность \thesisSpecialtyTwoNumber\ ---\par <<\thesisSpecialtyTwoTitle>>}
\fi

\vspace{0pt plus3fill} %число перед fill = кратность относительно некоторого расстояния fill, кусками которого заполнены пустые места
\textbf{\large{АВТОРЕФЕРАТ}}

\vspace{0pt plus2fill}

\large{\textbf{диссертации на соискание учёной степени}\par \textbf{\thesisDegree}}
\end{center}

\vspace{0pt plus13fill} %число перед fill = кратность относительно некоторого расстояния fill, кусками которого заполнены пустые места
{\centering\textbf{\thesisCity~--- \thesisYear}\par}

\newpage
% оборотная сторона обложки
\thispagestyle{empty}
Работа выполнена в {\thesisInOrganization}.

\vspace{0.008\paperheight plus3fill}
\noindent%
\begin{tabularx}{\textwidth}{@{}lX@{}}
    \ifdefined\supervisorTwoFio
    Научные руководители:   & \supervisorRegalia\par
                              \ifdefined\supervisorDead
                              \framebox{\textbf{\supervisorFio}}
                              \else
                              \textbf{\supervisorFio}
                              \fi
                              \par
                              \vspace{0.013\paperheight}
                              \supervisorRegalia\par
                              \ifdefined\supervisorTwoDead
                              \framebox{\textbf{\supervisorTwoFio}}
                              \else
                              \textbf{\supervisorTwoFio}
                              \fi
                              \vspace{0.013\paperheight}\\
    \else
    \textbf{Научный руководитель:}   & \ifdefined\supervisorDead
                              \framebox{\textbf{\supervisorFio}}
                              \else
                              \textbf{\supervisorFio  --- \supervisorRegalia}
                              \fi
                              \vspace{0.013\paperheight}\\
    \fi
    % Официальные оппоненты:  &
    % \ifnumequal{\value{showopplead}}{0}{\vspace{13\onelineskip plus1fill}}{%
    %     \textbf{\opponentOneFio,}\par
    %     \opponentOneRegalia,\par
    %     \opponentOneJobPlace,\par
    %     \opponentOneJobPost\par
    %     \vspace{0.01\paperheight}
    %     \textbf{\opponentTwoFio,}\par
    %     \opponentTwoRegalia,\par
    %     \opponentTwoJobPlace,\par
    %     \opponentTwoJobPost
    % \ifdefined\opponentThreeFio
    %     \par
    %     \vspace{0.01\paperheight}
    %     \textbf{\opponentThreeFio,}\par
    %     \opponentThreeRegalia,\par
    %     \opponentThreeJobPlace,\par
    %     \opponentThreeJobPost
    % \fi
    % }%
    % \vspace{0.013\paperheight} \\
    \ifdefined\leadingOrganizationTitle
    \textbf{Ведущая организация:}    &
    \ifnumequal{\value{showopplead}}{0}{\vspace{6\onelineskip plus1fill}}{%
        \textbf{\leadingOrganizationTitle}
    }%
    \fi
\end{tabularx}

\vspace{0.008\paperheight plus3fill}

Защита состоится \underline{\textbf{\defenseDate}} на заседании диссертационного совета \textbf{\defenseCouncilNumber}, созданного на базе федерального государственного автономного образовательного учреждения высшего образования <<Сколковский Инстиутут Науки и Технолгий>> (Сколтех)

\textbf{по адресу:} \defenseCouncilAddress.

\vspace{0.008\paperheight plus1fill}
С диссертацией можно ознакомиться в библиотеке \synopsisLibrary.

% \vspace{0.008\paperheight plus1fill}
% \noindent Отзывы на автореферат в двух экземплярах, заверенные печатью учреждения, просьба направлять по адресу: \defenseCouncilAddress, ученому секретарю диссертационного совета~\defenseCouncilNumber.

\vspace{0.008\paperheight plus3fill}
{Автореферат разослан \synopsisDate.}


\vspace{0.008\paperheight plus9fill}
\noindent%
\begin{tabularx}{\textwidth}{@{}%
>{\raggedright\arraybackslash}b{14em}@{}
>{\centering\arraybackslash}X
r
@{}}
    \small{\textbf{Ученый секретарь}}\par
    \textbf{\small{диссертационного совета}}
    &
    \ifnumequal{\value{showsecrsign}}{0}{}{%
        \includegraphics[width=2cm]{secretary-signature.png}%
    }%
    &
    \textbf{\small{\defenseSecretaryFio}}
\end{tabularx}
           % Титульный лист
\ifnumequal{\value{englishthesis}}{1}{
    \thispagestyle{empty}
\begin{center}
\thesisOrganizationEn
\end{center}

\noindent%
\begin{tabularx}{\textwidth}{@{}lXr@{}}%
    & & \textit{\textbf{As a manuscript}}\\
    % \IfFileExists{images/logo.pdf}{\includegraphics[height=0.5cm]{logo}}{\rule[0pt]{0pt}{2.5cm}}  & &
    \ifnumequal{\value{showperssign}}{0}{%
        \rule[0pt]{0pt}{1.5cm}
    }{
        \includegraphics[height=1.5cm]{personal-signature.png}
    }\\
\end{tabularx}

\vspace{0pt plus9fill} %число перед fill = кратность относительно некоторого расстояния fill, кусками которого заполнены пустые места
\begin{center}
\textbf {\large \thesisAuthorEn}
% \end{center}

\vspace{0pt plus1fill} %число перед fill = кратность относительно некоторого расстояния fill, кусками которого заполнены пустые места
% \begin{center}
\textbf {\large %\MakeUppercase
\thesisTitleEn}

\vspace{0pt plus1fill} %число перед fill = кратность относительно некоторого расстояния fill, кусками которого заполнены пустые места
{\large \textbf{Speciality: \thesisSpecialtyNumber.~\thesisSpecialtyTitleEn}}

\ifdefined\thesisSpecialtyTwoNumberEn
{\large Speciality \thesisSpecialtyTwoNumberEn\ "---\par <<\thesisSpecialtyTwoTitleEn>>}
\fi

\vspace{0pt plus3fill} %число перед fill = кратность относительно некоторого расстояния fill, кусками которого заполнены пустые места
\textbf{\large{DISSERTATION ABSTRACT}}

\vspace{0pt plus2fill}

\large{\textbf{of the dissertation for the Degree of} \par \textbf{\thesisDegreeEn}}
\end{center}

\vspace{0pt plus13fill} %число перед fill = кратность относительно некоторого расстояния fill, кусками которого заполнены пустые места
{\centering\textbf{\thesisCityEn~--- \thesisYear}\par}

\newpage
% оборотная сторона обложки
\thispagestyle{empty}
The dissertation was prepared at the {\thesisOrganizationEnNonTitle}.

\vspace{0.008\paperheight plus3fill}
\noindent%
\begin{tabularx}{\textwidth}{@{}lX@{}}
    \ifdefined\supervisorTwoFio
    Научные руководители:   & \supervisorRegalia\par
                              \ifdefined\supervisorDead
                              \framebox{\textbf{\supervisorFio}}
                              \else
                              \textbf{\supervisorFio}
                              \fi
                              \par
                              \vspace{0.013\paperheight}
                              \supervisorRegalia\par
                              \ifdefined\supervisorTwoDead
                              \framebox{\textbf{\supervisorTwoFio}}
                              \else
                              \textbf{\supervisorTwoFio}
                              \fi
                              \vspace{0.013\paperheight}\\
    \else
    \textbf{Scientific supervisor:}   & \ifdefined\supervisorDead
                              \framebox{\textbf{\supervisorFioEn}}
                              \else
                              \textbf{\supervisorFioEn  --- \supervisorRegaliaEn}
                              \fi
                              \vspace{0.013\paperheight}\\
    \fi
    % Official opponents:  &
    % \ifnumequal{\value{showopplead}}{0}{\vspace{13\onelineskip plus1fill}}{%
    %     \textbf{\opponentOneFioEn,}\par
    %     \opponentOneRegaliaEn,\par
    %     \opponentOneJobPlaceEn,\par
    %     \opponentOneJobPostEn\par
    %     \vspace{0.01\paperheight}
    %     \textbf{\opponentTwoFioEn,}\par
    %     \opponentTwoRegaliaEn,\par
    %     \opponentTwoJobPlaceEn,\par
    %     \opponentTwoJobPostEn
    % \ifdefined\opponentThreeFio
    %     \par
    %     \vspace{0.01\paperheight}
    %     \textbf{\opponentThreeFio,}\par
    %     \opponentThreeRegalia,\par
    %     \opponentThreeJobPlace,\par
    %     \opponentThreeJobPost
    % \fi
    % }%
    % \vspace{0.013\paperheight} \\
    % \ifdefined\leadingOrganizationTitleEn
    % \textbf{Leading organization:}    &
    % \ifnumequal{\value{showopplead}}{0}{\vspace{6\onelineskip plus1fill}}{%
    %     \textbf{\leadingOrganizationTitleEn}
    % }%
    % \fi
\end{tabularx}
\vspace{0.008\paperheight plus3fill}

The defense will take place on \underline{\textbf{\defenseDateEn}}~at~the~meeting of the Dissertation Council \textbf{\defenseCouncilNumber}, based at Skolkovo Institute of Science and Technology 

\textbf{address:} \defenseCouncilAddressEn.

\vspace{0.008\paperheight plus1fill}
The dissertation can be found in the \synopsisLibraryEn.

% \vspace{0.008\paperheight plus1fill}
% \noindent Reviews of the synopsis should be sent in two copies, stamped by the organization, to \defenseCouncilAddressEn, to the secretary of the dissertation council~\defenseCouncilNumberEn.

\vspace{0.008\paperheight plus3fill}
{The abstract was sent out on \synopsisDateEn.}


\vspace{0.008\paperheight plus9fill}
\noindent%
\begin{tabularx}{\textwidth}{@{}%
>{\raggedright\arraybackslash}b{14em}@{}
>{\centering\arraybackslash}X
r
@{}}
    \small{\textbf{Academic secretary}} \par \textbf{\small{of the 
    Dissertation council,}}\par
    \textbf{\small{\defenseSecretaryRegaliaEn}}
    &
    \ifnumequal{\value{showsecrsign}}{0}{}{%
        \includegraphics[width=2cm]{secretary-signature.png}%
    }%
    &
    \textbf{\small{\defenseSecretaryFioEn}}
\end{tabularx}

}{}
\chapter*{Аннотация}
Сочетание больших языковых моделей и графов знаний для ответов на вопросы нацелено на использование языковых навыков моделей и фактической точности графов. Однако языковые модели часто создают «галлюцинации». Данная диссертация решает эти проблемы, предлагая и тестируя новые методы для надежного объединения языковых моделей и графов знаний с целью улучшения точности, надежности и управляемости вопросно-ответных систем.

Сделаны два основных вклада. Первый — это отбор кандидатов в ответы по типу. Этот метод улучшает генерацию ответов, используя способность языковой модели предсказывать семантический тип ответа, даже если первоначальный фактический ответ неверен. Эта информация о типе, вместе с правилами типов из графа знаний, помогает фильтровать и переранжировать кандидатов. Эксперименты на таких наборах данных, как SimpleQuestions-Wikidata, RuBQ и Mintaka, показывают стабильный прирост Hits@1 для различных языковых моделей, превосходя стандартные модели.

Второй ключевой вклад — это система для контролируемого объединения с использованием переранжирования на основе подграфов. Этот метод повышает фактическую корректность ответов, сгенерированных языковыми моделями, проверяя несколько кандидатов по данным из подграфов графа знаний. Эти подграфы связывают сущности вопроса с кандидатами. Процесс включает извлечение подграфов, создание признаков из них и использование различных моделей ранжирования. Это последовательно улучшает выбор ответа. Для содействия исследованиям в этой области был создан набор данных ShortPathQA, предлагающий вопросы с готовыми подграфами графа знаний.

Практические системы с API-интерфейсами и инструментом визуализации подграфов показывают, что эти методы могут использоваться в реальных приложениях. Они служат примерами и инструментами для исследователей.

В итоге, данная диссертация дает новое понимание совместной работы языковых моделей и графов знаний. Она предлагает методы для создания более точных, надежных и понятных вопросно-ответных систем для графов знаний. Эти техники улучшают фактологическую основу и поддерживают создание более контролируемого искусственного интеллекта.
\chapter*{Abstract}

The most flexible and general approach to modeling random perturbations in constrained optimization is the use of Chance Constraints (CC). This type of constraint allows for the pre-setting of the probability of violating the initial constraints, thereby preventing excessive conservatism in the solution. Generally, CCs do not admit a closed-form expression, which prevents their direct use in numerical methods. To circumvent this limitation, various types of data-driven approximations have been proposed, notably the Scenario Approximation (SA). Despite theoretical guarantees for obtaining a feasibe solution with high probability, the required amount of data (scenarios) is high, leading to a high-dimensional optimization problem. This dissertation proposes methods and algorithms for evaluating the CC value and solving optimization problems with CCs, which are significantly less demanding in terms of the amount of data (scenarios) needed to obtain a feasible solution with high probability.

A method for evaluating the value of CC was developed using an optimization-statistical approach of adaptive importance sampling and demonstrated for the feasibility estimation of the current generation mode in the power grid. Two approaches to scenario selection for SA were proposed for the optimization with additive and multiplicative uncertainty. The basis of the approach is the identification of the region of redundant scenarios that are not representative in terms of constraint violations. Their efficiency is demonstrated for optimal power flow problem.

According to the research results, a significant improvement in the convergence rate of the variance of the estimate to the minimum was demonstrated for evaluating the value of CC, reducing dependence on the dimensionality of the problem to $O(\sqrt{\log K})$, where 
$K$ is the number of constraints. The previous results claim linear dependence  - $O(K)$. Numerical experiments showed that the proposed method is more stable in certain synthetic settings, compared to other modern methods and more efficient for electrical networks setups. For optimization with CC, it was theoretically shown that the number of required scenarios to obtain a feasible solution with high probability is reduced, and numerically, the required number of scenarios was reduced by an average of 2 times.
\include{Dissertation/contents}        % Оглавление
\ifnumequal{\value{contnumfig}}{1}{}{\counterwithout{figure}{chapter}}
\ifnumequal{\value{contnumtab}}{1}{}{\counterwithout{table}{chapter}}
% % \chapter*{Abstract}

% Apparently the GOST for dissertations does not include an abstract

\addcontentsline{toc}{chapter}{Introduction}
\chapter*{Introduction}

% Template and formatting:
% All Skoltech theses have an abstract
% One thesis had chapter summary at the end of each chapter. Looks like a good idea
% Introduction is apparently just a regular chapter

\input{common/characteristic}

\section*{Publications}

\paragraph{Thesis publications.} The thesis is based on the following three Q1 publications:

\begin{enumerate}
    \item \fullcite{lukashevich2021importance}
    \item \fullcite{lukashevich2021power}
    \item \fullcite{lukashevich2023importance}
\end{enumerate}

In all the above papers, the author was a principal contributor, who developed and implemented all listed algorithms, proved (in the last paper jointly with A. Bulkin) all supporting theorems and lemmas. 

\paragraph{Other publications by the author.} Beyond the aforementioned publications, the author during his MS and PhD research published a few other papers, including five Q1 papers and two A-level conferences, exploring mathematically related problems in other fields:
\begin{enumerate}
    \item \fullcite{shevchenko2023climate}
    \item \fullcite{morozov2023cmip}
    \item \fullcite{mitrovic2023gp}
    \item \fullcite{mitrovic2023fast}
    \item \fullcite{mitrovic2023data}
    \item \fullcite{dvurechensky2022hyperfast}
    \item \fullcite{agafonov2021accelerated}
    \item \fullcite{tanuishkina2024scirep}
\end{enumerate}

\section*{Acknowledgments} 
The dissertation was completed at the {\thesisOrganizationEn}.

I thank my advisor, Elena Gryazina, for providing guidance and assistance at all stages of my doctoral study. I would like to devote special and sincere gratitude to my former scientific advisor and long-term co-author Yury Maximov, who initally showed the research potential in the field of statistical methods in power system and supported me throughout the whole research started even before my graduate study period. I also thank my coauthors, Deepjyoti Deka, Mile Mitrovic, Petr Vorobev and Vyacheslav Gorchakov, for numerous insightful discussions. Last but not least, I thank my wife, my family and my friends for immense support during my scientific journey.

\chapter{Introduction}
\label{chap:Introduction}


%Begin by providing a brief overview of power systems to set the context. Explain why power systems are critical, touching upon their role in modern society, economic importance, and their influence on technological advancements.
\section{Background and context.}
Carbon-free electricity generation is one of the most vital global challenges for the next decades. Renewable energy sources, such as wind, hydro, and solar power generation, are increasingly demanded, accessible, and widely used in modern power grids due to their ecological and economic benefits. For instance, California’s renewable portfolio standard currently mandates that 33\% of retail electricity sales come from renewable resources, with targets set to rise to 60\% by 2030 and 100\% by 2045. However, the integration of renewable energy generation introduces significant volatility and uncertainty to power systems, posing numerous challenges for operators. 

In 2020, electricity production accounted for approximately 25\% of greenhouse gas emissions in the USA, making the integration of renewable energy a critical strategy for reducing emissions. Nonetheless, the increased variability in power generation and disturbances associated with renewables compromise grid security and challenge traditional operation and planning policies. Integrating renewable energy sources also aligns with the United Nations' sustainable development goals by promoting affordable and clean energy while enhancing energy security and resilience. Unfortunately, these benefits come with substantial challenges to grid optimization and control policies due to the significant uncertainties introduced by renewable energy sources.

%Example:
%"Power systems are the backbone of modern society, providing the necessary energy to drive economic activities, industrial processes, and daily life. As the demand for reliable and efficient energy continues to grow, the complexity and significance of power systems have increased, necessitating advanced research and development in this field."
\section{Problem statement.}
%2. Problem Statement
%Identify the specific problem or gap in the current knowledge that your research addresses. This should be a concise statement that clearly defines the issue your thesis will tackle.
Various algorithms have been developed to ensure grid reliability, utilizing methods ranging from machine learning to analytical and sampling-based approaches. Machine learning methods leverage historical data on weather, renewable generation, and grid operating parameters to estimate risks but are impractical for real-time operation due to their reliance on large datasets and extensive data collection times. Analytical approximation methods, which compute overload probabilities through integrals, often overestimate risks, particularly in rare events, compromising their practical efficiency. Sampling-based algorithms, such as Monte Carlo (MC) simulations, provide valuable alternatives for assessing reliability but struggle with the performance in evaluating rare, severe disturbances due to uniform exploration of fluctuation spaces.

The Optimal Power Flow (OPF) problem, crucial for determining economically optimal power generation levels under given constraints, has several extensions to address uncertainty in power generation and consumption. Robust and chance-constrained formulations are popular, with the former assuming bounded uncertainty and the latter requiring high-probability satisfaction of security constraints. The Joint Chance-Constrained Optimal Power Flow (JCC-OPF) problem bounds the probability of security constraint failures but is computationally hard even under linear security limits and Gaussian uncertainty. Tractable convex approximations often yield conservative solutions unsuitable for practical operations. Scenario and Sample Average Approximations, which replace stochastic elements with deterministic inequalities, can handle non-Gaussian uncertainties but may require a large number of samples, complicating their application in large-scale grids.

Additionally, the discrete-time dynamic chance-constrained OPF problem addresses optimal generation set-points over sequential timestamps, incorporating ramp-up and ramp-down constraints to manage the rate of power output changes. Automatic Generation Control (AGC) aids in efficient power dispatch, yet solving the chance-constrained problem for arbitrary distributions and joint technical limits remains computationally infeasible. Data-driven approximations, such as Scenario Approximation (SA) and Sample Average Approximation (SAA), although effective, are often computationally prohibitive when high accuracy is needed, necessitating extensive scenario reduction studies. This complexity highlights the need for improved methods to handle the uncertainties and operational challenges posed by the integration of renewable energy sources into power systems.

Summing up, modern power system are influenced by various uncertainty sources and require modern data-driven and data-efficient methods to, firstly, estimate reliability of the current power system state, secondly, reliably control the conventional generators to reach the most economically efficient state, simultaneously satistying demand and meeting technical constraints.

%Example:
%"Despite significant advancements in power systems, challenges remain in improving the efficiency and stability of power grids, particularly with the integration of renewable energy sources. This research aims to address these challenges by exploring new circuit designs and control strategies."
\section{Research objectives.}

This research aims to develop advanced methods for improving the efficiency and accuracy of reliability assessments and optimization in power systems under uncertainty. We propose an adaptive importance sampling method to estimate the risk of reliability constraints violation more efficiently. This method utilizes physical information to create a mixture of distributions for sampling and employs convex optimization to iteratively adjust the weights of the mixture. By incorporating importance sampling, we aim to reduce the complexity and enhance the accuracy of scenario approximations for chance-constrained optimal power flow. This approach generates more informative samples, resulting in an optimization problem with fewer constraints. Additionally, we propose an A-priori Reduced Scenario Approximation (AR-SA) method, which integrates data-driven scenario approximation techniques to reduce the number of samples required while maintaining solution reliability. This method seeks to provide theoretical guarantees for the feasibility of solutions in joint chance-constrained dynamic optimal power flow problems. Our research extends existing importance sampling techniques and contributes to more effective and practical solutions for managing uncertainties in power system operations.

\section{Significance of the Study}

The research holds significant importance as it addresses critical challenges in modeling renewable energy sources higly penetrated power systems, which is vital for achieving carbon-free electricity generation. By developing advanced methods such as adaptive importance sampling and A-priori Reduced Scenario Approximation (AR-SA), the research aims to improve the efficiency and accuracy of reliability assessments and optimization under uncertainty. These methods contribute to overcoming limitations in current approaches, such as impracticality for real-time operations, overestimation of risks, and computational infeasibility in large-scale power grids. The outcomes of this research are expected to enhance the operational reliability and economic efficiency of power systems, facilitating the integration of renewable energy sources. This aligns with global sustainability goals by promoting cleaner energy solutions, reducing greenhouse gas emissions, and enhancing energy security and resilience. Ultimately, the research contributes to the advancement of power system technologies, supporting the global transition towards a more sustainable energy future.
% 4. Significance of the Study
% Explain the importance of your research and its potential impact. Discuss how your findings could contribute to the field of power systems and their broader implications.

% Example:
% "This study is significant as it aims to contribute to the development of more efficient and stable power systems. By addressing the challenges associated with renewable energy integration, the research has the potential to enhance the sustainability and reliability of future power grids, supporting the global transition towards greener energy solutions."

% 5. Scope and Limitations
% Define the scope of your research, including what will and will not be covered. This helps to set clear boundaries and manage the expectations of your readers.

% Example:
% "The scope of this research is limited to the development and analysis of circuit designs and control strategies within the context of renewable energy integration. It does not cover other aspects of power systems such as economic analysis or policy implications."

% 6. Structure of the Thesis
% Provide an outline of the subsequent chapters and briefly describe their content. This helps readers understand the organization of your thesis and the logical flow of your research.

% Example:
% "The thesis is organized into six chapters. Chapter 2 reviews the relevant literature on power systems and renewable energy integration. Chapter 3 details the theoretical framework and methodologies used in the research. Chapter 4 presents the development of the proposed circuit designs. Chapter 5 discusses the control strategies for renewable energy integration. Chapter 6 evaluates the performance of the proposed solutions through simulations and experimental results. Finally, Chapter 7 concludes the thesis with a summary of findings and suggestions for future research."

% 7. Summary
% Conclude the introduction with a brief summary that reiterates the importance of your research and sets the stage for the detailed exploration in the following chapters.

% Example:
% "In summary, this thesis aims to address critical challenges in power systems, particularly in the integration of renewable energy sources. By developing innovative circuit designs and control strategies, this research seeks to enhance the efficiency and stability of power grids, contributing to the advancement of sustainable energy solutions. The following chapters will delve into the theoretical foundations, methodologies, and empirical findings that support these objectives."

% Additional Tips
% Be Clear and Concise: Avoid unnecessary jargon and complex sentences. Aim for clarity and brevity to ensure your introduction is accessible to a broad audience.
% Engage the Reader: Start with a compelling statement or fact to capture the reader's interest.
% Cite Relevant Literature: Support your statements with references to key studies and authoritative sources in the field.
% By following these guidelines, you can craft an effective introduction chapter that provides a strong foundation for your PhD thesis on power systems.

\section{Summary.}
The integration of renewable energy sources into modern power systems presents both significant opportunities and challenges. While these sources are crucial for achieving carbon-free electricity generation and meeting global sustainability goals, their inherent variability and uncertainty pose risks to grid reliability and stability. This research identifies the limitations of current methods for managing these uncertainties, such as machine learning, analytical approximations, and sampling-based algorithms, particularly in the context of the Optimal Power Flow (OPF) problem and its extensions. In response, we propose novel methods, including adaptive importance sampling and A-priori Reduced Scenario Approximation (AR-SA), to enhance the efficiency and accuracy of reliability assessments and optimization in power systems under uncertainty. By addressing these challenges, the research aims to improve the operational reliability and economic efficiency of power systems, facilitating the successful integration of renewable energy sources and supporting the global transition towards a sustainable energy future.



    % Введение
% \ifnumequal{\value{contnumfig}}{1}{\counterwithout{figure}{chapter}
% }{\counterwithin{figure}{chapter}}
% \ifnumequal{\value{contnumtab}}{1}{\counterwithout{table}{chapter}
% }{\counterwithin{table}{chapter}}



%%% Thesis Chapters based on new structure
\ifnumequal{\value{contnumfig}}{1}{\counterwithout{figure}{chapter}}{\counterwithin{figure}{chapter}}
\ifnumequal{\value{contnumtab}}{1}{\counterwithout{table}{chapter}}{\counterwithin{table}{chapter}}

\chapter*{Introduction}
\label{chap:introduction}
\addcontentsline{toc}{chapter}{Introduction}

\textbf{Background and Relevance of the Work.}
For a long time, a key goal in artificial intelligence has been to make machines that can understand and answer human questions using large collections of structured knowledge. Knowledge Graphs (KGs) are good for this because they can show how facts are connected, which helps give exact answers. The arrival of Large Language Models (LLMs) has changed this area a lot. LLMs are trained on huge amounts of text and are very good at understanding and creating language. They also store a lot of factual knowledge inside themselves. However, when we use LLMs for Knowledge Graph Question Answering (KGQA), they have some problems. A big problem is that they can create answers that sound good but are wrong. This is often called hallucination~\cite{lin-etal-2022-truthfulqa, DBLP:conf/emnlp/RobertsRS20, DBLP:journals/csur/JiLFYSXIBMF23, DBLP:journals/corr/abs-2401-01313}. Also, it is hard to understand how LLMs come up with their answers, so it's difficult to check if the answers are based on facts.


Now that LLMs are used so much, KGs are even more important as a source of true, organized knowledge. LLMs are good at creating natural text and understanding general meanings. But they are not so good at handling facts that change often. They also have trouble adding new information without messing up the knowledge they already have. For example, LLMs alone often cannot answer comparative questions correctly if they do not have access to structured information, as shown in our work on the CAM 2.0 system~\cite{DBLP:conf/coling/ShalloufHSVMPBN24}. Our research in~\cite{pletenev-etal-2025-much} looked at the problems of adding new facts into LLMs using Low-rank Adaptation (LoRA). This study showed that using LoRA to add new knowledge can be harmful. The LLM might become worse at general question answering. This happens especially if the new knowledge is mostly about certain things. The LLM might then often give those overrepresented answers and be less likely to say when it's not sure. These results show that we really need good ways to combine the text-making skills of LLMs with the facts from KGs. We should not just try to update the LLM's own knowledge. This dissertation works on this problem. It offers and tests new methods for mixing LLMs and KGs better, to make KGQA systems more trustworthy.

\textit{Note: In some portion of this document (20-30\% of the entire text); Artificial Intelligence assistant, particularly Generative AI, has been used to improve, rephrase, shorten, or summarize the content. The technologies used include Gemini 2.5 Pro, Grok and Perplexity. All generated content was manually reviewed and edited to ensure accuracy and coherence.}

\textbf{Dissertation Objectives.}
This dissertation wants to create and test new ways to combine Large Language Models with Knowledge Graphs. The goal is to make Question Answering systems that are more accurate, dependable, and controllable. The main objectives are:
\begin{itemize}
    \item To make better answer candidates in KGQA. We can do this by using the LLMs' ability to understand meaning, together with specific type rules from KGs. This includes creating a method called Answer Candidate Type (ACT) Selection. This method predicts the likely semantic type of an answer. It then uses this type to filter and improve candidate answers (Chapter~\ref{chap:act_selection} has more details).
    \item To facilitate controllable and standardized community research on the fusion of Knowledge Graphs (KGs) and Large Language Models (LLMs) for KGQA, through the development and public release of the \texttt{ShortPathQA} dataset. This resource provides pre-computed KG subgraphs, allowing researchers to concentrate on fusion methodologies rather than complex KG processing tasks (as introduced in Chapter~\ref{sec:controllable_fusion:dataset}).
    \item To make LLM-generated answers more factually correct and controllable. We can do this by creating a system to re-rank candidate answers. This system uses structural proof from KG subgraphs. This includes finding important subgraphs that connect question entities to LLM-generated candidates. It also includes getting different features from these subgraphs and using various rankers to choose candidates that have stronger KG-based proof (Chapter~\ref{chap:controllable_fusion} has more details).
\end{itemize}

\textbf{Scientific Novelty.}
The new scientific ideas in this dissertation are in creating and testing new ways to mix LLMs and KGs. These ways help solve important problems in KGQA:
\begin{enumerate}
    \item \textbf{New Answer Candidate Typing and Selection:} The Answer Candidate Type~(ACT) Selection method (Chapter~\ref{chap:act_selection}, \cite{DBLP:journals/corr/abs-2310-07008}) is a new way to improve KGQA. It is special because it combines an LLM's skill to guess the semantic type of an answer with the clear type information from a KG (for example, Wikidata's P31 `instance of' property). This type-based checking of LLM-generated candidates works like a strong filter and scoring tool. It greatly improves the quality of candidate answers, even if the LLM first gives the wrong answer but gets its type right.
    \item \textbf{Controllable Fusion via Subgraph Reranking:} This work proposes a novel framework for improving the factuality of LLM outputs by reranking answer candidates based on structural evidence from KG subgraphs (Chapter~\ref{chap:controllable_fusion}, \cite{DBLP:journals/corr/abs-2310-02166}). The novelty lies in:
    \begin{itemize}
        \item The systematic extraction of relevant KG subgraphs connecting question entities to multiple LLM-generated answer candidates.
        \item The comprehensive feature engineering from these subgraphs, encompassing graph-theoretic metrics (e.g., PageRank, Katz centrality, density), textual features (question-answer concatenation), and various Graph2Text (G2T) sequence representations (Deterministic, T5-based, and GAP-based).
        \item The application and comparative analysis of diverse ranking models (from simple regression to neural rankers like MPNet) to effectively utilize these multi-modal subgraph features for promoting factually grounded answers.
    \end{itemize}
    \item \textbf{Creation of a Specialized Evaluation Resource (ShortPathQA):} The development and publication of the \texttt{ShortPathQA} dataset (Chapter~\ref{sec:controllable_fusion:dataset}, \cite{DBLP:conf/nldb/SalnikovSPQA25}) is a significant novel contribution. This dataset directly supports the research on Controllable Fusion via Subgraph Reranking presented in this dissertation and aims to foster further community exploration of such methods. As the first QA resource offering pre-computed KG subgraphs, \texttt{ShortPathQA} allows researchers to focus on the core tasks of subgraph reasoning and reranking. This isolates these tasks from the complexities of upstream entity linking and large-scale KG processing, thereby facilitating more targeted evaluation and development of controllable fusion techniques.
\end{enumerate}

\textbf{Structure and volume of the dissertation.}
This dissertation is structured to systematically present these contributions across four chapters, with a logical progression from foundational concepts through individual methodologies to integrated system implementations and system demonstrations. Introduction and Chapter~\ref{chap:related_work} provides a detailed overview of the background and relevance of the work. Chapter~\ref{chap:act_selection} introduces the Answer Candidate Type Selection method. Chapter~\ref{chap:controllable_fusion} introduces the Controllable Fusion via Subgraph Reranking method. Chapter~\ref{chap:system_demos} introduces the system demonstrations for the ACT Selection and tools for supporting Controllable Fusion methods.

\textbf{Theoretical and Practical Significance.}
The research presented in this dissertation holds both theoretical and practical significance for the fields of Natural Language Processing and Artificial Intelligence, particularly in the domain of Knowledge Graph Question Answering.

Theoretically, this work contributes to a deeper understanding of how the complementary strengths of LLMs (which are good at semantic understanding and fluency) and KGs (which provide structured factual knowledge and verifiability) can be synergistically combined. It explores specific mechanisms for this fusion, moving beyond simple retrieval augmentation towards more nuanced methods of candidate refinement and evidence-based reranking. The investigation into type-based filtering (ACT Selection) sheds light on the implicit semantic knowledge captured by LLMs and how it can be explicitly leveraged in conjunction with KG schema for improved reasoning. The study of subgraph-based reranking contributes to theories of evidence-based reasoning in QA, demonstrating how structural path information in KGs can serve as a robust signal for factuality and control in LLM outputs. Furthermore, the creation of the \texttt{ShortPathQA} dataset facilitates more focused theoretical explorations into subgraph reasoning by providing a standardized benchmark.

Practically, the methodologies developed, particularly ACT Selection and subgraph-based reranking, offer practical pathways to improve the factual accuracy and reliability of QA systems. This is crucial for real-world applications where incorrect information can have significant consequences. By making LLM outputs more controllable and grounded in verifiable KG facts, this research contributes to building more trustworthy AI systems. The ability to trace an answer back to KG evidence, as facilitated by subgraph analysis, enhances interpretability. The ACT Selection method offers a resource-efficient strategy to boost the performance of even smaller LLMs, making high-quality KGQA more accessible when computational resources are constrained. The \texttt{ShortPathQA} dataset provides a valuable, ready-to-use resource for other researchers and practitioners, lowering the barrier to entry for developing and evaluating LLM-KG fusion techniques without the need for extensive KG infrastructure setup. Finally, the system demonstrations (Chapter~\ref{chap:system_demos}), including API endpoints and visualization tools, provide tangible proofs-of-concept and reusable components for building advanced KGQA applications.

\textbf{Research Methodology.}
The research in this dissertation employs a multifaceted methodology combining theoretical development with empirical experimentation and system implementation.
\begin{enumerate}
    \item \textbf{Literature Review and Problem Formulation:} The work begins with a comprehensive review of existing research in KGQA, LLMs, and LLM-KG fusion strategies (Chapter~\ref{chap:related_work}) to identify gaps and formulate precise research questions.
    \item \textbf{Method Development - ACT Selection (Chapter~\ref{chap:act_selection}):}
    \begin{itemize}
        \item Observation of LLM behavior regarding answer type prediction, even in cases of factual error.
        \item Design of a pipeline that:
            \item Generates initial answer candidates using LLMs (e.g., T5, BART with Diverse Beam Search).
            \item Employs multilingual entity linking (e.g., mGENRE) to connect question and candidate entities to Wikidata.
            \item Infers the expected answer type using the LLM and aggregates type information from KG `instance-of` relations.
            \item Ranks candidates using a weighted scoring mechanism incorporating type scores, KG neighborhood scores, LLM generation scores, and question-property similarity.
    \end{itemize}
    \item \textbf{Method Development - Controllable Fusion via Subgraph Reranking (Chapter~\ref{chap:controllable_fusion}):}
    \begin{itemize}
        \item Hypothesis that KG path-based evidence can improve LLM answer factuality.
        \item Design of a reranking pipeline that:
            \item Generates multiple answer candidates from LLMs using Diverse Beam Search.
            \item Extracts KG subgraphs (shortest paths) connecting question entities to each candidate answer using Wikidata.
            \item Engineers diverse features from these subgraphs:
                \item Graph features: number of nodes/edges, cycles, bridges, average shortest path, density, Katz centrality, PageRank.
                \item Text features: concatenation of question and answer, encoded with MPNet.
                \item Graph2Text (G2T) features: using Deterministic linearization, T5-based G2T, and GAP-based G2T models, often with question context and answer highlighting.
            \item Employs and compares various reranking models: semantic (MPNet cosine similarity), regression (Linear, Logistic), gradient boosting (CatBoost), and neural (MPNet with regression head).
    \end{itemize}
    \item \textbf{Dataset Creation (ShortPathQA - Chapter~\ref{sec:controllable_fusion:dataset}):}
    \begin{itemize}
        \item Addressing the need for a focused benchmark for subgraph-based reasoning.
        \item Collection of questions from Mintaka (filtered for entity answers) and manual curation of new complex questions.
        \item Generation of answer candidates using LLMs.
        \item Unified shortest path subgraph extraction from Wikidata for all question-candidate pairs.
        \item Publication of the dataset with questions, candidates, and pre-computed subgraphs.
    \end{itemize}
    \item \textbf{Experimental Evaluation:}
    \begin{itemize}
        \item Utilization of established KGQA datasets (SQWD, RuBQ, Mintaka) and the newly created ShortPathQA.
        \item Fine-tuning of LLMs (T5, BART, spaCy NER models) on relevant training splits.
        \item Evaluation metrics appropriate for the tasks: Hits@1 for factoid accuracy in ACT Selection and ShortPathQA classification; Hits@N (N=1,2,3) and F1-score for reranking performance in controllable fusion and ShortPathQA.
        \item Ablation studies to assess the contribution of individual components and feature sets.
        \item Error analysis to understand model behavior and the corrective capabilities of the proposed methods.
        \item Comparison against strong baselines, including standalone LLMs (e.g., ChatGPT) and existing KGQA systems.
    \end{itemize}
    \item \textbf{System Implementation and Demonstration (Chapter~\ref{chap:system_demos}):}
    \begin{itemize}
        \item Development of practical system pipelines (baseline Seq2Seq, M3M, ACT Selection) using Python, FastAPI, PyTorch, HuggingFace Transformers, spaCy, etc.
        \item Creation of a subgraph visualization tool using web technologies (HTML, JavaScript, D3.js) to support research and provide interactive exploration.
    \end{itemize}
\end{enumerate}

\textbf{Validation of Research Results and Reliability.}
The research findings presented in this dissertation are validated through rigorous empirical evaluation on multiple standard benchmark datasets and newly created resources. The reliability of the results is ensured by:
\begin{itemize}
    \item \textbf{Standardized Datasets and Metrics:} For the ACT Selection method (Chapter~\ref{chap:act_selection}), evaluations were performed on three Wikidata-based one-hop KGQA datasets: SimpleQuestions-Wikidata (SQWD), RuBQ (English translations), and a subset of Mintaka (one-hop English questions). The primary metric was Hits@1, which directly measures the accuracy of the top-ranked answer.
    \item For the controllable fusion methods (Chapter~\ref{chap:controllable_fusion}), experiments were conducted on the Mintaka dataset (excluding count and yes/no questions) and the newly introduced ShortPathQA dataset. Evaluation metrics included Hits@N (N=1, 2, 3) to assess the reranker's ability to position the correct answer within the top N candidates, and F1-score for binary classification tasks on ShortPathQA.
    \item \textbf{Comparison with Baselines:} The performance of the proposed methods was consistently compared against various baselines.
        \item In ACT Selection, comparisons included the base Text-to-Text models (T5 and BART variants, both zero-shot and fine-tuned) without ACT, specialized KGQA systems like QAnswer and KEQA, and large conversational models like ChatGPT.
        \item In controllable fusion, baselines included initial LLM predictions (T5-Large-SSM, T5-XL-SSM, Mistral, Mixtral), random ranking, semantic reranking, and simpler regression-based models before evaluating more complex neural rankers. For ShortPathQA, comparisons included zero-shot LLMs (GPT-4o, LLaMA3-8b-Instruct) and supervised models (MPNet, fine-tuned LLaMA3-8b-Instruct).
    \item \textbf{Ablation Studies:} Comprehensive ablation studies were conducted (e.g., Table~\ref{tab:act_selection:ablation_study} for ACT Selection, and analyses in Chapter~\ref{chap:controllable_fusion} for feature set combinations) to assess the individual contributions of different components of the proposed frameworks (e.g., different scoring mechanisms, feature types). This helps to isolate the impact of novel elements.
    \item \textbf{Error Analysis:} Qualitative error analysis was performed (e.g., in Chapter~\ref{chap:act_selection} and for G2T methods in Chapter~\ref{chap:controllable_fusion}) to understand specific instances where the proposed methods succeed or fail, providing deeper insights beyond aggregate metrics. For example, ACT Selection demonstrated an ability to predict correct answer types in 94\% of instances for T5-Large-SSM on SQWD, a significant improvement over the 61\% baseline. The G2T analysis highlighted T5's superior factual accuracy over GAP in preserving entity information.
    \item \textbf{Cross-Model and Cross-Dataset Evaluation:} The proposed methods were evaluated across different LLM architectures (T5, BART, Mistral, Mixtral) and sizes, and across multiple datasets with varying characteristics, demonstrating the robustness and generalizability of the findings. For example, ACT Selection consistently improved performance across all tested base models and datasets. The subgraph reranking methods also showed consistent Hits@1 improvements across different LLMs.
    \item \textbf{Statistical Significance and Reproducibility:} While not always explicitly stated as statistical significance tests, the consistent and considerable margins of improvement over baselines across multiple setups suggest meaningful results. The publication of code (e.g., M3M pipeline, ShortPathQA dataset and code) and detailed experimental setups (hyperparameters, model versions) supports reproducibility. For the \texttt{ShortPathQA} dataset, a manual test set was curated with multiple annotators, achieving a Cohen's kappa of 0.81 for question type and an average quality score of 4.81/5, ensuring data quality.
\end{itemize}
These measures collectively ensure that the conclusions drawn are well-supported by empirical evidence and that the proposed methods offer reliable improvements in KGQA.

\textbf{Approbation of the Work and Publications.}
The research presented in this dissertation has been disseminated through peer-reviewed publications in international conferences and journals, and through participation in shared tasks and system demonstrations. The key contributions have been presented to and validated by the scientific community.

The publications by the author that form the basis of or are related to this dissertation are:
\begin{enumerate}
    \item \textbf{[Scopus, Conference Proceedings, CORE A*]} Anton Razzhigaev*, \textbf{Mikhail Salnikov*}, Valentin Malykh, Pavel Braslavski, and Alexander Panchenko. 2023. A System for Answering Simple Questions in Multiple Languages. In Proceedings of the 61st Annual Meeting of the Association for Computational Linguistics (Volume 3: System Demonstrations), pages 524–537, Toronto, Canada. Association for Computational Linguistics.
    \item \textbf{[Conference Proceedings]} \textbf{Mikhail Salnikov}, Maria Lysyuk, Pavel Braslavski, Anton Razzhigaev, Valentin A. Malykh, and Alexander Panchenko. 2023. Answer Candidate Type Selection: Text-To-Text Language Model for Closed Book Question Answering Meets Knowledge Graphs. In Proceedings of the 19th Conference on Natural Language Processing (KONVENS 2023), pages 155–164, Ingolstadt, Germany. Association for Computational Linguistics. 
    \item \textbf{[Scopus, Conference Proceedings, CORE C]} \textbf{Mikhail Salnikov}, Hai Le, Prateek Rajput, Irina Nikishina, Pavel Braslavski, Valentin Malykh, and Alexander Panchenko. 2023. Large Language Models Meet Knowledge Graphs to Answer Factoid Questions. In Proceedings of the 37th Pacific Asia Conference on Language, Information and Computation, pages 635–644, Hong Kong, China. Association for Computational Linguistics.
    \item \textbf{[Conference Proceedings]} Andrey Sakhovskiy*, \textbf{Mikhail Salnikov*}, Irina Nikishina, Aida Usmanova, Angelie Kraft, Cedric Möller, Debayan Banerjee, Junbo Huang, Longquan Jiang, Rana Abdullah, Xi Yan, Dmitry Ustalov, Elena Tutubalina, Ricardo Usbeck, and Alexander Panchenko. 2024. TextGraphs 2024 Shared Task on Text-Graph Representations for Knowledge Graph Question Answering. In Proceedings of TextGraphs-17: Graph-based Methods for Natural Language Processing, pages 116–125, Bangkok, Thailand. Association for Computational Linguistics.
    \item \textbf{[Scopus]} \textbf{Mikhail Salnikov*}, Andrey Sakhovskiy*, Irina Nikishina, Aida Usmanova, Angelie Kraft, Cedric Möller, Debayan Banerjee, Junbo Huang, Longquan Jiang, Rana Abdullah, Xi Yan, Elena Tutubalina, Ricardo Usbeck, and Alexander Panchenko. (2025). ShortPathQA: A Dataset for Controllable Fusion of Large Language Models with Knowledge Graphs. In \textit{Natural Language Processing and Information Systems - 30th International Conference on Applications of Natural Language to Information Systems, NLDB 2025}.
    \item \textbf{[Scopus, Conference Proceedings, CORE B]} Ahmad Shallouf*, Hanna Herasimchyk*, \textbf{Mikhail Salnikov*}, Rudy Alexandro Garrido Veliz*, Natia Mestvirishvili*, Alexander Panchenko, Chris Biemann, and Irina Nikishina. 2024. CAM 2.0: End-to-End Open Domain Comparative Question Answering System. In Proceedings of the 2024 Joint International Conference on Computational Linguistics, Language Resources and Evaluation (LREC-COLING 2024), pages 2657–2672, Torino, Italia. ELRA and ICCL.
    \item \textbf{[Scopus, WoS]} Maria Lysyuk, \textbf{Mikhail Salnikov}, Pavel Braslavski, and Alexander Panchenko. (2024). Konstruktor: A Strong Baseline for Simple Knowledge Graph Question Answering In International Conference on Applications of Natural Language to Information Systems, pp. 107-118. Cham: Springer Nature Switzerland, 2024.
    \item \textbf{[Scopus, Conference Proceedings]} Dmitrii Iarosh, Alexander Panchenko, and \textbf{Mikhail Salnikov}. (2025). On Reducing Factual Hallucinations in Graph-to-Text Generation Using Large Language Models. In \textit{Proceedings of the Workshop on Generative AI and Knowledge Graphs (GenAIK)}.
    \item \textbf{[Scopus, Conference Proceedings, CORE A]} Sergey Pletenev, Maria Marina, Daniil Moskovskiy, Vasily Konovalov, Pavel Braslavski, Alexander Panchenko, and \textbf{Mikhail Salnikov}. 2025. How Much Knowledge Can You Pack into a LoRA Adapter without Harming LLM?. In Findings of the Association for Computational Linguistics: NAACL 2025, pages 4309–4322, Albuquerque, New Mexico. Association for Computational Linguistics.
\end{enumerate}

\textbf{Author's Personal Contribution.}

Author's personal contribution to the research presented in this dissertation is as follows:
\begin{itemize}
    \item \textbf{ACT Selection Method Development (Chapter~\ref{chap:act_selection}):} The author was fully responsible for the conceptualization, implementation, and experimental validation of the Answer Candidate Type Selection pipeline, including the design of the four-component scoring mechanism.
    \item \textbf{Subgraph Reranking Framework (Chapter~\ref{chap:controllable_fusion}):} The author led the full development of the controllable fusion methodology and the design of the experiments. The implementation and experimental design for the Graph-To-Text (G2T) models were developed by the author's advisor with significant contributions from Hai Le, Dmitrii Iarosh, and Ivan Lazichny. Olga Tsymboi and Egor Cheremiskin implemented experiments with the Mistral and Mixtral models.
    \item \textbf{ShortPathQA Dataset Creation (Chapter~\ref{sec:controllable_fusion:dataset}):} The author had full responsibility for the dataset design and the subgraph generation pipeline. The annotation framework and question curation for the manual portion of the dataset were handled by Andrey Sakhovskiy and Irina Nikishina. Andrey Sakhovskiy conducted experiments with encoder-only baselines.
    \item \textbf{System Demonstration Creation (Chapter~\ref{chap:system_demos}):} The author had full responsibility for creating the system demonstrations. Dmitrii Iarosh assisted with the development of the subgraph visualization tool and related DevOps tasks.
\end{itemize}

    % Chapter 1: Introduction (Replaces old intro.tex)
\chapter{Background and Related Work}
\label{chap:related_work}

% 2.1 Large Language Models for Question Answering
\section{Large Language Models for Question Answering}
\label{sec:rw_llms_qa}
% Comment: Provide necessary background on LLMs (architectures like Transformers, 
% pre-training/fine-tuning paradigms) and their application to various QA formats 
% (e.g., extractive, abstractive, open-domain, closed-book). Discuss limitations 
% relevant to factuality.


% 2.2 Knowledge Graphs and Semantic Technologies
\section{Knowledge Graphs and Semantic Technologies}
\label{sec:rw_kgs}
% Comment: Explain foundational concepts of KGs (nodes, edges, triples), 
% representation standards (RDF, OWL), query languages (SPARQL), and prominent public 
% KGs (Wikidata, DBpedia).


% 2.3 Knowledge Graph Question Answering (KGQA)
\section{Knowledge Graph Question Answering (KGQA)}
\label{sec:rw_kgqa}
% Comment: Review traditional KGQA methods that predate heavy LLM integration (e.g., 
% semantic parsing to SPARQL, embedding-based methods). You could briefly mention 
% "Konstruktor" here as an example of a simple baseline KGQA system, positioning it 
% relative to the more complex LLM-integrated approaches you focus on later.


% 2.4 Integrating Large Language Models and Knowledge Graphs
\section{Integrating Large Language Models and Knowledge Graphs}
\label{sec:rw_llm_kg_integration}
% Comment: This is a key section. Survey the existing literature on combining LLMs and 
% KGs. Categorize different approaches (e.g., KG-augmented retrieval, using LLMs to 
% generate KG queries, using KGs to verify/rerank LLM outputs, joint modeling). 
% Position your work within this landscape, highlighting how your methods (especially 
% the controllable fusion approach) differ from or improve upon prior work. Mention 
% the "How Much Knowledge Can You Pack..." paper here to contrast external KG 
% integration with internal knowledge packing via adapters like LoRA.


% 2.5 Related QA Tasks: Multilingual and Comparative QA
\section{Related QA Tasks: Multilingual and Comparative QA}
\label{sec:rw_related_qa}
% Comment: Briefly discuss related work specifically in multilingual QA (context for "A 
% System for Answering Simple Questions...") and comparative QA (context for "CAM 
% 2.0..."). Explain how insights or techniques from these areas relate to your core 
% focus on factoid QA with KGs.     % Chapter 2: Background and Related Work
\chapter{Methods for Fusing LLMs and KGs for Enhanced Factuality}
\label{chap:methods}

% Comment: This chapter details your core methodological contributions, drawing heavily 
% from "Answer Candidate Type Selection...", "Large Language Models Meet KGs...", 
% and the methodological aspects of "ShortPathQA...

This chapter delves into specific methodologies developed to fuse the generative capabilities of Large Language Models (LLMs) with the structured factual knowledge encoded in Knowledge Graphs (KGs). As discussed in Chapter~\ref{chap:introduction}, while LLMs excel at text generation and comprehension, they often struggle with factual accuracy and are prone to generating plausible-sounding but incorrect information, a phenomenon often referred to as hallucination~\cite{lin-etal-2022-truthfulqa, DBLP:conf/emnlp/RobertsRS20}. Fusing LLMs with external KGs presents a compelling strategy to mitigate these limitations by grounding model outputs in verified factual structures~\cite{DBLP:journals/tkde/PanLWCWW24}. This integration process, broadly termed LLM-KG fusion, aims to leverage the strengths of both paradigms: the fluency and contextual understanding of LLMs (parametric knowledge) and the precision and reliability of KGs (non-parametric knowledge)~\cite{DBLP:conf/acl/MallenAZDKH23}.

Several approaches exist for this fusion. Some methods focus on encoding graph structure directly, using techniques like Graph Convolutional Networks (GCNs) for tasks like graph-to-text generation. However, recent work suggests that LLMs themselves, with appropriate prompting or fine-tuning, can often outperform specialized GCN architectures, particularly in handling the nuances of language and reducing factual errors~\cite{iarosh-etal-2025-reducing, DBLP:conf/ijcai/0001LW0S0Y24}. Another direction involves injecting knowledge more directly into the LLM parameters using techniques like adapters, attempting to distill relational facts into the model itself~\cite{DBLP:journals/corr/abs-2002-01808}. A prominent and flexible approach involves using the KG as an external knowledge source during inference, often retrieving relevant subgraphs or paths and incorporating this information into the LLM's prompt or generation process~\cite{DBLP:conf/emnlp/KnowledgeAugmented}.

This chapter focuses specifically on methods developed throughout our research that fall primarily within this latter category of leveraging external KG information to guide LLM generation for factoid QA. We introduce two distinct techniques: first, Answer Candidate Type (ACT) Selection~\cite{DBLP:journals/corr/abs-2310-07008}, which uses the LLM's understanding of semantic types combined with KG constraints; and second, a method utilizing explicit KG path information~\cite{DBLP:journals/corr/abs-2310-02166}, {\color{red}TODO: Add ShortPathQA, and Jouranl PAPER HERE} which provides direct factual evidence to the LLM. These methods form the foundation for the work detailed in the subsequent sections, aiming to create more controllable and factually reliable QA systems.


\section{Answer Candidate Type Selection - find the correct answer by extracting the type of the answer from LLM generated candidates}
\label{sec:methods_type_selection}

While Large Language Models (LLMs) demonstrate impressive capabilities in various natural language tasks, their application to factoid Question Answering (QA) in a closed-book setting is often marred by factual inaccuracies and hallucinations \cite{DBLP:conf/emnlp/RobertsRS20, lin-etal-2022-truthfulqa}. Interestingly, however, even when an LLM fails to produce the correct factual answer, it often exhibits an understanding of the question's underlying intent and the semantic \textbf{type} of the expected answer. For instance, when asked "What is the official language of Brazil?", an LLM might incorrectly respond "Spanish" but correctly identify that the answer should be a \textbf{language}. Similarly, for "Who directed the movie Inception?", it might hallucinate a director's name but still understand the answer should be a \textbf{person} (specifically, a film director).

This observation suggests a potential avenue for improving factuality. If we can reliably predict the expected answer type, this information can serve as a valuable constraint. Knowing the type is particularly useful when combined with the main entity mentioned in the question. The correct answer entity within a Knowledge Graph (KG) is often located semantically "close" to the question entity, connected via specific relations, and, crucially, belongs to the expected semantic type. For example, knowing the question entity is "Brazil" and the answer type is `dbo:Language` significantly narrows down the plausible candidates within the KG compared to an unconstrained search or generation process.

This section presents our proposed approach, Answer Candidate Type~(ACT) Selection. We propose a universal approach to selecting the correct answer in the KGQA task by using any pre-trained large language model to generate answer candidates and to infer the type of expected answer. The answer candidate type selection pipeline shown in Figures~\ref{fig:act_selection:pipeline} and \ref{fig:act_selection:pipeline_example}.

\begin{figure}[h]
    \centering
    \includegraphics[width=0.85\textwidth]{act_selection/pipeline.png}
    \caption{ACT Selection Pipeline: (1) Put question to sequence to sequence language model to generate answer candidates, (2) Extract type from candidates, (3) Extract entities from the questions and query one-hop neighbors of the entities in the KG, (4) Filter candidates by type, (5) Select the best candidate.}
    \label{fig:act_selection:pipeline}
\end{figure}

\begin{figure}[h]
    \centering
    \includegraphics[width=0.95\textwidth]{act_selection/pipeline_example.png}
    \caption{The Answer Candidate Type (ACT) Selection pipeline for Knowledge Graph Question Answering (KGQA). The process combines a sequence-to-sequence model with knowledge graph-based entity linking and scoring to identify the correct answer, "Konami," as the publisher of "Neo Contra."}
    \label{fig:act_selection:pipeline_example}
\end{figure}

\subsection{Initial Answer Candidate List Generation} 
To increase the diversity of the generated results, we use Diverse Beam Search~\cite{DBLP:journals/corr/VijayakumarCSSL16-diverse-beam-search} to generate an initial list of answer candidates $C$. It often leads to a better exploration of the search space by ensuring that alternative answers are considered. It will discussed in more detail in Section~\ref{sec:methods_kg_path_fusion:expansion_of_generated_candidates}. We define the types of entities using the Wikidata property \texttt{instance\_of}~(P31). Note that an entity can be of multiple types. Finally, the initial list of answer candidates is used in the Answer Candidate Typing and the Candidate Scorer with the mined candidates. 

\subsection{Answer Candidate Typing} \label{act}
We rank all types by their frequency in the initial list of answer candidates. 
After that, we merge the top-$K$ most frequent types and similar types to the final list $T$.
Types similarity is calculated as cosine similarity between Sentence-BERT~\cite{reimers-2019-sentence-bert} embeddings of respective labels. The final types are defined as the ones where similarity is greater than a threshold.

A similar aggregation method using hypernyms (also known as ''is-a''  or ''instance-of'' relations) was used in the past to label clusters of words senses in distributional models~\cite{biemann2013text}: distributionally similar words share common hypernym and top common hypernyms are surprisingly good labels for sense clusters. The analogy in our method is that language models appear to produce a list of distributionally similar candidates.

\subsection{Entity Linking}
To enrich the list of candidates, we add all one-hop neighbors of the entities found in the question. For that, we use the fine-tuned spaCy Named Entity Recognition~(NER)\footnote{\url{https://spacy.io}.}

Based on a comprehensive review of state-of-the-art Named Entity Recognition (NER) systems \cite{vajjala-balasubramaniam-2022-really}, we evaluated the top three approaches: spaCy\footnote{\url{https://spacy.io}}, Stanza\footnote{\url{https://stanfordnlp.github.io/stanza/}}, and SparkNLP\footnote{\url{https://nlp.johnsnowlabs.com}}. Our analysis revealed that pre-trained NER models performed poorly on the Simple Question Wikidata (SQWD)~\cite{SQ_WD} dataset, with entity detection failure rates ranging from 64\% to 88\%. Among these systems, spaCy demonstrated the best performance, leading us to select its standard configuration\footnote{\url{https://spacy.io/usage/training/}} for further fine-tuning. The implementation of this pipeline required two critical pre-processing steps. First, we needed to identify the entity span to feed into the algorithm. This necessitated first extracting entity labels and their corresponding redirects, then locating these labels within the question text to determine spans. For entities without exact matches in the question, we employed fuzzy search techniques\footnote{\url{https://pypi.org/project/fuzzywuzzy/}}. Second, spaCy's training process requires entity type tags (e.g., PERSON for "Elon Musk", ORG for "Tesla" - following the BIO tagging scheme), which were not provided in the original datasets. We initially assigned the PERSON tag universally across all entities. Subsequent experiments with more precise entity type tagging did not yield significant improvements.
After extensive evaluation, we ultimately selected mGENRE \cite{decao2021multilingual} as our entity linking solution, which demonstrated superior performance compared to the NER-based approaches. mGENRE functions as an end-to-end entity linking system that eliminates the need for separate entity detection and disambiguation steps. Its multilingual autoregressive entity linking capabilities provided better accuracy and robustness across diverse question formulations in our experiments, making it a practical choice for our knowledge graph question answering pipeline.

\subsection{Candidates Scorer}
Finally, we calculate four scores for each entity from the answer candidate and rank them based on the weighted sum of the scores.
The scores are as follows:
% \begin{itemize}
    % \item
    \textbf{(1)~Type score} represents the size of the intersection between the set of types extracted from the answer candidates and the selected answer types. It is weighted by the number of selected answer types: $$S_\textrm{type}~=~\frac{|\textrm{Candidates' Types} \cap T|}{|T|}.$$

    % \item
\textbf{(2)~Forward one-hop neighbors score} $S_\textrm{neighbour}$ is  assigned 1 if the candidate is among the neighbors of the question entities, and 0 otherwise.

    % \item
\textbf{(3)~Text-to-Text answer candidate score} is determined by the rank of the candidate in the initial list $C$ generated by the Text-to-Text model divided by the size of the list: $$S_\textrm{t2t}~=~\frac{C.\textsc{}{index}(\textrm{Candidate})}{|C|}.$$

    % \item
\textbf{(4)~Question-Property Similarity score} $S_\textrm{property}$ measures the cosine similarity between the embeddings of the relevant property and the entire question. We employ Sentence-BERT~\cite{reimers-2019-sentence-bert} to encode the question, following a similar approach used for the Answer Candidate Type module.
% \end{itemize}
The four scores are calculated for each entity and then are combined to generate a final score that determines the entity's ranking. The answer with the highest weighted sum of scores in the candidate list is selected as the final answer:
%
$$S_\textrm{final} = S_\textrm{type} + S_\textrm{neighbour} + S_\textrm{t2t} + S_\textrm{property}.$$

% The weights for this sum can be estimated using logistic regression on a sample dataset in a future works.

% Text-to-Text models, despite occasionally providing incorrect answers, still yield answers of the correct type that can be utilized within the KGQA pipeline. This is demonstrated in the example depicted in Figure~\ref{fig:pipeline_example}.

% TODO: Добавить больше текста - заключение для метода 

% 3.2 Controllable Fusion using Knowledge Graph Paths
\section{Controllable Fusion using Knowledge Graph Paths}
\label{sec:methods_kg_path_fusion}

While constraining the answer type, as discussed in Section~\ref{sec:methods_type_selection}, provides a valuable signal for improving factuality, another powerful approach involves leveraging the explicit relational structure within the Knowledge Graph more directly. This section introduces methods based on the core idea of using KG subgraphs - a small part of the KG that contains the entities from the question, answer candidate generated by the LLM, and the shortest paths between them, drawing upon the work presented in \cite{DBLP:journals/corr/abs-2310-02166} and subsequent refinements focused on reranking 
{\color{red} TODO: add cite SWJ Reranking Answers of Large Language Models with Knowledge Graphs}.

The central motivation is to move beyond implicit knowledge within the LLM and provide explicit, verifiable evidence trails from the KG. Instead of solely relying on the LLM to generate a factually correct answer, we can use paths connecting entities mentioned in the question to potential answer entities within the KG as a strong indicator of factual consistency. This approach offers greater controllability and interpretability, as the KG path itself serves as evidence supporting a given answer.

The general pipeline for this fusion strategy, illustrated in Figure~\ref{fig:methods_kg_path_fusion:big_pipe}, involves several stages. First, relevant entities are identified within the input question. Second, candidate answers might be generated by an LLM, or potential answer entities are retrieved from the KG, focusing on entities connected to the question entities via relatively short paths. Crucially, features are extracted from these connecting paths (or the surrounding subgraph). These features capture information about the path structure and relations involved. Finally, these subgraph based features are used by a scoring or reranking model to evaluate the correctness of each candidate answer, ultimately selecting the answer best supported by the explicit structure of the Knowledge Graph. This section will detail the specific mechanisms developed for implementing this subgraph based fusion method.

\begin{figure}[htb]
    \centering
    \includegraphics[width=\columnwidth]{kg_path_fusion/new_paper_big_pipeline.pdf}
    \caption{The proposed method for reranking language model answers with KGs. The method includes subgraph extraction, features extraction, and various ranker approaches.}
    \label{fig:methods_kg_path_fusion:big_pipe}
\end{figure}
  
% In this section, we examine each component of the process in detail.
%  Firstly, we generate answer candidates using various LLMs and generate subgraphs, as discussed in Subsection~\ref{sec:subgraph_extract}. Next, in Subsection~\ref{sec:subgraph_features}, we will look at which attributes are used to rank responses.

\subsection{Answer Candidate Generation} \label{sec:methods_kg_path_fusion:expansion_of_generated_candidates}
As the subgraph extraction protocol requires answer candidates, we need a source of distinct answer candidates for each question, but most LLM approaches for QA, such as the one presented by \cite{DBLP:conf/coling/SenAS22-mintaka}, typically use \texttt{Greed Search} and evaluate the top-1 answer, it is important to note that the correct answer may not always be the top candidate. For example, the fine-tuned T5-XL-SSM~\cite{DBLP:conf/emnlp/RobertsRS20} model achieved higher Mean Reciprocal Rank (MRR) scores for our task, indicating that re-ranking could improve the top-1 results. 
However, even when using \texttt{Classical Beam Search}, the output is often minor variations of a single sequence, which may not generate enough unique answer candidates for the Question Answering task. Similar to the approach a little bit described in Section~\ref{sec:methods_type_selection}, we use Diverse Beam Search~\cite{DBLP:journals/corr/VijayakumarCSSL16-diverse-beam-search} to generate an initial list of answer candidates.

The formula~\ref{eq:methods_kg_path_fusion:diverse_beam_search} involves splitting the set of beams at time $t$ into $g$ disjointed subsets $Y_{[t]}^g$, and then selecting the candidate with the highest diversity penalty, which is calculated as the sum of a diversity penalty function $\Theta(y_{b,[t]}^g)$ over all candidates in the subset. Additionally, a dissimilarity term is included, which is calculated as the sum of a dissimilarity function $\Delta(y_{b,[t]}^g, Y_{[t]}^h)$ over all previous subsets $Y_{[t]}^h$ up to time $g-1$. The dissimilarity term is weighted by a parameter $\lambda_g$. This formula is used to optimize the selection of answer candidates in a computationally efficient manner.

\begin{equation}
    \begin{aligned}
        Y_{[t]}^g = \quad & \underset{y_1^{g}, \dots, y_{B\prime}^g \in Y_t^g} {\text{argmax}} \quad \underbrace{\sum_{b \in [B\prime]} \Theta(y_{b, [t]}^g)}_{\text{diversity penalty}} \\ 
        & + \underbrace{\sum_{h=1}^{g-1} \lambda_g \Delta(y_{b,[t]}^g, Y_{[t]}^h)}_{\text{dissimilarity term}},
    \end{aligned} 
    \label{eq:methods_kg_path_fusion:diverse_beam_search}
\end{equation}

% We apply {Diverse Beam Search} to the following LLMs: {T5-large-ssm}, {T5-XL-ssm}, {Mistral}, and {Mixtral} with $200$ beams, $20$ beam groups, and a $0.1$ diversity penalty. We extend our previous research~\cite{DBLP:conf/paclic/SalnikovLRNBMP23-originalpaper} by fine-tuning the proposed T5-like models and comparing them to more recent state-of-the-art models like Mistral and Mixtral, which should make our research more applicable to real-world use cases. T5-large-SSM and T5-XL-SSM were reported to be state-of-the-art both in the original Mintaka paper and our previous work, serving as a good baseline for comparison in this study.

% To finetune the T5-like models, we first train them on English questions for $10000$ steps, following the protocols outlined in the original Mintaka paper~\cite{DBLP:conf/coling/SenAS22-mintaka}. For the more state-of-the-art Mistral and Mixtral, we finetune with LoRA and train on English questions by generating the answer candidates with ``\textit{Answer as briefly as possible without additional information. [Question]}''. However, for the T5-like models, despite adhering to these protocols, we could not achieve the reported Hits@1 accuracy in the original paper. Despite this challenge, the main focus of the study is on the reranking aspect of the pipeline. Therefore, this paper's primary contribution is improving our fine-tuned models.

\subsection{Subgraph Extraction} \label{sec:subgraph_extract}
The main backbone of our approach is the procedure of subgraph extraction. We rely on the information conveyed in the relationships between question-answer pairs to improve the reranking of LLM generations. To further investigate how this relationship can improve performance, we employ a subgraph extraction algorithm that generates a KG's subgraph containing entities relevant to each question-answer pair and the shortest paths between them that contain relevant properties/relationships. 
% We extract various features that can be used for reranking in addition to the subgraphs. This section presents the subgraph extraction algorithm, the features derived from the subgraphs, and our reranking approaches.

\begin{figure}[htb]
    \centering
    \includegraphics[width=\columnwidth]{kg_path_fusion/ssp_to_sub.pdf}
    \caption{The proposed method for reranking language model answers with KGs. The method includes subgraph extraction, features extraction, and various ranker approaches.}
    \label{fig:methods_kg_path_fusion:subgraph_construction_example}
\end{figure}

For each question-answer candidate pair, the desired subgraph $G$ is mathematically defined as an induced subgraph of the Knowledge Graph. Thus, given our shortest paths from $e_i~\rightarrow~A$, where $e_i$ - entity extracted from question and $A$ - Answer. We can use the following Listing~\ref{alg:methods_kg_path_fusion:sub_extract} to extract $G$. Let us define $H$ as the set of all distinct nodes within our shortest paths $P_i$. We want to preserve all edges between the nodes within $H$. For all question-answer pairs, our objective is to retain the relationship between our question entities $E$ and answer candidate entity $A_i$. The process is schematically depicted at Figure~\ref{fig:methods_kg_path_fusion:subgraph_construction_example}.

\begin{ListingEnv}[p]
    \centering % Center the listing if desired
    \caption{Subgraph Extraction Algorithm} 
    \label{alg:methods_kg_path_fusion:sub_extract} 
    \begin{lstlisting}[basicstyle=\fontsize{10pt}{12pt}\selectfont\ttfamily] % Smaller font for code Require: entities, candidate

paths = []
For entity in entities:
    shortest_paths = get_shortest_path(entity, candidate)
    paths.extend(shortest_paths)

H = set of unique nodes in paths

G = new Graph()
Add nodes from H to G

For unique_node in H:
    unique_node_neighbors = get_neighbors(unique_node)
    For neighbor_node in unique_node_neighbors:
        If neighbor_node in H:
            Add edge (unique_node, neighbor_node) to G

Return G
    \end{lstlisting}
\end{ListingEnv}


\subsection{Features based on Extracted Subgraphs} \label{sec:methods_kg_path_fusion:subgraph_features}
After extracting subgraphs for all answer candidates of our LMs, we use all possible useful features for reranking. Referring to our previous study, we mainly focused on a simple text representation of the extracted subgraphs to rank our answer candidates. Thus, in this study, we propose extracting as many useful features as possible and analyzing each feature's importance in this reranking problem. We have divided the features into the following main categories: graph, text, and Graph2Text sequence features. 

\subsubsection{Graph Features} \label{sec:methods_kg_path_fusion:graph_features}
With our extracted subgraphs and their corresponding answer candidate, we seek to use the relationship from the subgraphs to classify the correct answer candidate. As the first simple baseline, we utilize graph features consisting of simple numerical subgraph statistics. We hypothesize that subgraphs with the correct answer will be less ``complex'' than subgraphs with the incorrect answer candidate. Therefore, we would want the graph features to convey the complexity of the respective subgraph. With a clear objective in mind, we experiment with the following graph features:  

\begin{itemize}
    \item \textbf{Number of nodes and edges}: basic statistics of the nodes and edges of graph $G$.
    \item \textbf{Number of cycles}: a cycle of graph $G$ is a non-empty path that starts from a given node and ends at the same node. 
    \item \textbf{Number of bridges}: a bridge of graph $G$ is an edge, where its deletion increases the number of connection components. 
    \item \textbf{Average shortest path}: the average of each shortest path between the question entity and the answer entity. 
    \item \textbf{Density}: measurement of the density of a graph, where the number of edges in a dense graph is close to the maximal number of edges (each pair of nodes is connected by an edge). The density $d$ for the graph $G$ is formulated as $d = \frac{m}{n(n-1)}$, where $n$ is the number of nodes and $m$ is the number of edges in $G$.
   \item \textbf{Katz centrality}~\cite{katz1953new}: measurement of the importance (or ``centrality'' - how ``central'' a node is in the graph) of a specific node $i$ in a graph $G$. The Katz centrality for node $i$ of graph $G$ is formulated as $x_i = \alpha \sum_{j} A_{ij} x_j + \beta$, where $A$ is the adjacency matrix of graph $G$ with eigenvalues $\lambda$, $\beta$ is the parameter that controls the initial centrality, and $\alpha < \frac{1}{\lambda_{\max}}$. 
    \item \textbf{PageRank}~\cite{page1999pagerank}: a popular algorithm used by Google to rank web pages in the search query by counting the number and quality of links to a page to determine an estimate of its importance. In graph theory, the ``web pages'' and ``links'' are synonymous with nodes and edges. 
\end{itemize}

\noindent We hypothesize that these features may provide ranker models with insights into the complexity of the respective subgraphs.


\subsubsection{Text Features} \label{sec:methods_kg_path_fusion:text_features} The ablation study showcased the importance of including the question within the text representation of the subgraph. Therefore, besides the simple graph features, we want to emphasize each question/answer pair without using extracted subgraphs. Thus, the text features represent the concatenation between the question and answer, separated by a semicolon --- ``\texttt{;}''. To use this simple concatenation for all ranker approaches, we encode the string using the MPNet\footnote{\url{https://huggingface.co/sentence-transformers/all-mpnet-base-v2}} embedding model~\cite{DBLP:conf/nips/Song0QLL20}.


\subsubsection{Graph2Text Sequence Features} \label{sec:methods_kg_path_fusion:g2t_seq} 
Given the vast amount of data contained in Knowledge Graphs, it is essential to convert this information into natural language to facilitate understanding and accessibility. Converting a graph into text, known as KG-to-text or Graph2Text, has demonstrated notable success in various applications~\cite{DBLP:journals/corr/abs-2309-11206}. Therefore, when generating text from a Knowledge Graph, it is crucial to analyze the underlying graph structure carefully to ensure accurate translation.

Without an obvious way of incorporating the question within the subgraphs, relying purely on the subgraphs to rerank is ineffective~\cite{DBLP:conf/paclic/SalnikovLRNBMP23-originalpaper}. Therefore, we address this issue by further exploration of different KG-to-text methods. The main objective is experimenting with various techniques to represent the extracted subgraphs more explicitly. For this type of textual feature, we researched and developed three methods for representing subgraphs as a text, including \textbf{Graph2Text Deterministic,  Graph2Text T5, and  Graph2Text GAP}.

Firstly, we employ the \textbf{Graph2Text Deterministic} approach, the most straightforward text linearization approach. In simple terms, the subgraphs are unraveled by their matrix representation. Firstly, to linearize, we convert the subgraph into its binary adjacency matrix. Let us call it matrix $A$. 
Given $n$ nodes in the subgraph, the resulting matrix's dimension will be $n \times n$. The matrix's element $[i, j]$ represents the existence of an edge between a node with index $i$ and a node with index $j$. Then, we replace the edges in the matrix with the edge label and call it $A'$. 
Adjacency matrices are typically implemented with graphs with numeric weights. The weights were string representing the relationship between our nodes. Thus, we represented the existence of an edge with $1$, then replaced in the edge with the string relationship.
%Let us call this adjacency matrix with edge information . 
Lastly, we unravel $A'$ row by row to produce our final sequence and add the triple (node\_from, edge, node\_to) to our final sequence. Listing~\ref{alg:methods_kg_path_fusion:sub2seq} summarizes the aforementioned steps. 

\begin{ListingEnv}[p]
    \centering % Center the listing if desired
    \caption{Subgraphs to Sequence - Graph2Text Deterministic} 
    \label{alg:methods_kg_path_fusion:sub2seq} 
    \begin{lstlisting}[basicstyle=\fontsize{10pt}{12pt}\selectfont\ttfamily] % Smaller font for code

Require: Subgraph G
Ensure: Text representation of subgraph Seq

adj_matrix = get_adjacency_matrix(G)
Seq = ""
# Assuming adj_matrix is n x n, where n is number of nodes
# and indices correspond to node IDs
For i in range(number_of_nodes(G)): 
    For j in range(number_of_nodes(G)):
        # Assuming 0 indicates no edge, non-zero indicates an edge
        If adj_matrix[i][j] != 0: 
            # Assuming G allows lookup by index i, j
            node_i_label = get_node_label(G, i) 
            node_j_label = get_node_label(G, j)
            # Get edge info (label/type) between node i and j
            edge_info = get_edge_info(G, i, j) 
            # Append triple to sequence string, separated by e.g., semicolon
            Seq += node_i_label + " " + edge_info + " " + node_j_label + "; " 
Return Seq
    \end{lstlisting}
\end{ListingEnv}

For the remaining two text linearization approaches, \textbf{Graph2Text T5} and \textbf{Graph2Text GAP}, we employ more complex neural-based models trained on the WebNLG~2.0 dataset~\cite{DBLP:conf/acl/GardentSNP17}. This dataset consists of instances, where each includes a Knowledge Graph from DBpedia~\cite{DBLP:conf/semweb/AuerBKLCI07} and a target text comprising one or more sentences that describe the graph. The test set is divided into partitions of seen (DBpedia categories present in the training set) and unseen (DBpedia categories not present in the training set). The statistics of this hand-crafted and human-verified dataset are described in detail in Table~\ref{tab:methods_kg_path_fusion:webnlg_label}. 

\begin{table}
    \centering
    \caption{Statistics of the WebNLG 2.0 parallel knowledge graph-to-text dataset.}
    
    \begin{subtable}[t]{0.48\textwidth}
    \centering
    \begin{tabular}{lr}
        \toprule
        \textbf{Entities} & 2,730 \\
        \textbf{Relations} & 354 \\
        \textbf{Triples} & 81,927 \\
        \bottomrule
    \end{tabular}
    \caption{Knowledge Graph statistics. Total number of KG components, number of tokens in the narratives.}
    \label{tab:kg_stats}
    \end{subtable}
    \hfill
    \begin{subtable}[t]{0.48\textwidth}
    \centering
    \begin{tabular}{lr}
        \toprule
        \textbf{Total} & 623,902 \\
        \textbf{Unique} & 8,075 \\
        \textbf{Entity} & 60\% \\
        \bottomrule
    \end{tabular}
    \caption{Texts statistics. The percentage of text entities represents the portion of the text that includes entity labels.}
    \label{tab:methods_kg_path_fusion:webnlg_label}
    \end{subtable}
\end{table}

The idea behind the \textbf{Graph2Text T5} approach is to extract informative and useful features from KGs using pre-trained text-to-text LMs. With the impressive capabilities of pretrained LMs in the text-to-text generation task, we seek to replicate such results in the graph-to-text scope. Our idea is built upon the analogous algorithm discussed in~\cite{DBLP:journals/corr/abs-2007-08426}. The authors tackle the graph-to-text generation task in this work with two popular text-to-text pre-trained LMs, BART and T5. These models have an encoder-decoder architecture, which makes them well-suited for conditional text generation tasks. To adapt these models for the graph-to-text task, the authors continue pre-training BART and T5 using the following approaches:

\begin{enumerate}
    \item Language Model Adaptation (LMA): the models are trained on reference texts that describe graphs, following the BART and T5 pre-training strategies.
    \item Supervised Task Adaptation (STA): the models are trained on pairs of graphs and their corresponding texts collected from the same or a similar domain as the target task --- graph-to-text in this case. 
\end{enumerate}

Building on the STA approach via T5 and WebNLG~2.0, we obtain graph-to-text sequences by first converting the graph into a sequence of tokens through linearization. We use the string ``convert the [graph] to [text]:'' to acquire this linearised sequence. This output sequence is then fed into the input sequence for the T5 model tuned on WebNLG~2.0. 
% For tuning Graph2Text T5 approach, we use the following hyperparameters: \textit{learning rate}: $1e^{-3}$, \textit{batch size}: 4, \textit{gradient accumulation steps}: 32, and \textit{Adam optimizer}. 

The \textbf{Graph2Text GAP} approach is based on the current state-of-the-art graph-to-text task, GAP, built on BART~\cite{DBLP:conf/coling/ColasAW22-GAP}. 
The main idea of GAP is a fully graph-aware encoding combined with the coverage of pre-trained LMs. The GAP KG-to-text framework fuses graph-aware elements into existing pre-trained LMs, capturing the advantages brought forth by both model types. The architecture of this solution consists of two main components:

\begin{enumerate}
\item \textbf{Global Attention}: to capture the graph's global semantic information, the graph’s components are first encoded using an LM. This allows the model to leverage the lexical coverage of pre-trained LMs.
\item \textbf{Graph-aware Attention}: to attend to and update the representations of entities, relations, or both, a topological-aware graph attention mechanism was introduced, which includes entity and relation type encoding.
\end{enumerate}

Applying the work of GAP, we first linearize the input graph into a text string by creating a sequence of all triples in the KG, interleaved with tokens that separate each triple and the triple's components (head, relation, and tail). Then, we use a transformer encoder to obtain vector representations. The first module in each transformer layer acts as a Global Attention and captures the semantic relationships between all tokens. Moreover, we use a Graph-aware Attention module to capture the sparse nature of adjustment in a graph and apply it to entity and relation vectors from word vectors. By proposing this flexible framework, where graph-aware components can be interchanged, the current architecture aims to generate coherent and representative text descriptions of the KG. Like the Graph2Text T5 approach, we pretrain the model on the WebNLG~2.0 dataset and get the final predictions through the fine-tuned model. 
% For finetuning the Graph2text GAP approach, we use the following hyperparameters: \textit{learning rate}: $2e^{-5}$, \textit{batch size}: 16; \textit{beam size}: 5, \textit{Adam Optimizer}, 50 \textit{nodes}, and 60 \textit{relations}.


% In this research, we introduce the more complex neural-based graph-to-text approach to explore further the reranking capabilities of the textual representation of our extracted subgraphs. The initial rudimentary text linearization approach has already achieved state-of-the-art Hits@1. We look for a more complete case study on reranking the text linearization with these two neural-based linearization approaches. To further digest the two methods, a comparison between the Graph2Text T5 and Graph2Text GAP sequences can be seen in the table ~\ref{tab:example_of_predictions_label}. Additionally, for better visualization, we implement a web application that automatically applies the T5 and GAP approaches to the desired subgraph, discussed in detail in Section ~\ref{sec:webapp}.
 

% \input{tabs/graph2text_examples}
% All three variation of Graph2Text Sequence features are further encoded with MPNet embedding model~\cite{DBLP:conf/nips/Song0QLL20}, discussed more in \ref{appx:detailed_results}. Moreover, motivated by our previous research, we employ context and highlight these Graph2Text sequences, discussed further in \ref{hl_context}.


% 3.3 Experimental Design and Baselines
\section{Experimental Design and Baselines}
\label{sec:methods_experimental_design}
% Comment: Describe the experimental setup used to evaluate the methods in 3.1 and 3.2. 
% Specify the datasets used (you might briefly introduce the motivation for ShortPathQA 
% here, leading into the next chapter), the evaluation metrics focused on factuality 
% and accuracy, and the baseline models (including perhaps zero-shot/few-shot LLMs, 
% and simpler KGQA systems).


% 3.4 Results and Analysis: Demonstrating Factuality Improvements
\section{Results and Analysis: Demonstrating Factuality Improvements}
\label{sec:methods_results}
% Comment: Present the quantitative results comparing your methods against baselines. 
% Crucially, analyze *why* your methods work, using examples and error analysis to 
% show how KG fusion mitigates specific failure modes of standalone LLMs (e.g., 
% hallucinations, factual errors). Discuss the trade-offs of each method.          % Chapter 3: Methods (Previously foundations, now incorporates type selection and reranking/path fusion)
\include{Dissertation/chapter_4_shortpathqa_benchmark} % Chapter 4: ShortPathQA Benchmark (Previously chapter_5)
\chapter{ShortPathQA: A Benchmark for Knowledge Graph Question Answering}
\label{chap:shortpathqa}

\section{Introduction}
\label{sec:shortpath:intro}
Building on the methods for answer candidate generation (Chapter~\ref{chap:candidate_generation}) and controllable fusion using knowledge graph paths (Chapter~\ref{chap:controllable_fusion}), this chapter introduces ShortPathQA, a novel benchmark for directly evaluating knowledge graph question answering capabilities. While the previous chapters focused on techniques to enhance LLM-generated answers with knowledge graphs, ShortPathQA explores a complementary approach that focuses on extracting precise answers directly from knowledge graph paths.

\section{The ShortPathQA Approach}
\label{sec:shortpath:method}
% TODO: Describe the methodology from the NLDB paper.

\section{Dataset Creation}
\label{sec:shortpath:dataset}
% TODO: Detail your work on creating the ShortPathQA dataset.

\section{Experimental Setup}
\label{sec:shortpath:setup}
% TODO: Describe the experiments conducted (baselines, metrics).

\section{Results and Analysis}
\label{sec:shortpath:results}
% TODO: Present the full experimental evaluation results.
% TODO: Compare performance characteristics with other potential approaches.

\section{Discussion}
\label{sec:shortpath:discussion}
% TODO: Discuss the strengths (e.g., precision) and weaknesses (e.g., coverage, flexibility) of this method.

\section{Chapter Summary}
\label{sec:shortpath:summary}
% TODO: Summarize the main contributions and findings regarding ShortPathQA.       % Chapter 5: System Demos (Previously chapter_8)
\chapter{On the Limits of Internalizing Knowledge: Insights from LoRA}
\label{chap:lora_limits}

% Comment: This chapter incorporates the findings from "How Much Knowledge Can You 
% Pack...", positioning it as an exploration complementary to the main KG fusion 
% theme.

% 6.1 Knowledge Integration Strategies: External vs. Internal
\section{Knowledge Integration Strategies: External vs. Internal}
\label{sec:lora_external_vs_internal}
% Comment: Revisit the idea of different ways to provide knowledge to LLMs. Contrast 
% the *external* approach (using KGs, as explored in Chapters 3-5) with *internal* 
% approaches like fine-tuning or parameter modification using methods like LoRA.


% 6.2 Investigating Knowledge Capacity in LoRA Adapters
\section{Investigating Knowledge Capacity in LoRA Adapters}
\label{sec:lora_capacity_investigation}
% Comment: Describe the study from "How Much Knowledge Can You Pack...". Explain the 
% methodology for "packing" knowledge into LoRA adapters and for measuring the 
% trade-off between acquired knowledge and the preservation of the base LLM's general 
% abilities.


% 6.3 Findings and Implications for KG-LLM Approaches
\section{Findings and Implications for KG-LLM Approaches}
\label{sec:lora_implications}
% Comment: Present the key results regarding the capacity limits of LoRA for storing 
% knowledge. Discuss the implications: Do these findings suggest inherent limitations 
% to solely relying on internalizing knowledge within LLM parameters, especially for 
% vast, dynamic factual information? Argue how these results might strengthen the case 
% for hybrid approaches that leverage external, structured knowledge sources like KGs 
% for robust factuality. 

\chapter{System Demonstrations and Implementation}
\label{chap:system_demos}

\section{Introduction}
\label{sec:demo:intro}
This chapter showcases the practical implementations of the methods discussed in earlier chapters. Moving from theoretical foundations to real-world applications, we present system demonstrations that illustrate how the fusion of Large Language Models (LLMs) and Knowledge Graphs (KGs) can be deployed to address factoid question answering tasks. These implementations serve not only as proof-of-concept for our methodological contributions but also as usable tools for researchers and potential end-users in the field.

\section{System for Answering Simple Questions}
\label{sec:demo:simple_qa}
% TODO: Describe the demo system from the ACL Demo paper.
% TODO: Explain its architecture and functionality.
% TODO: Connect it to the core methods discussed earlier.
% TODO: Mention your role in its development.

\section{Web Application for Knowledge Graph Path Visualization}
\label{sec:demo:kg_path_viz}
% TODO: Describe the web application for visualizing knowledge graph paths.
% TODO: Explain how it helps users understand the paths connecting entities.
% TODO: Discuss implementation details and technologies used.

\section{Reranking Demonstration Tool}
\label{sec:demo:reranking}
% TODO: Describe the demonstration tool for reranking answers.
% TODO: Explain how it incorporates the controllable fusion techniques.
% TODO: Showcase example usage and interface design.

\section{Integration with External Systems}
\label{sec:demo:integration}
% TODO: Discuss how these components can be integrated with other systems.
% TODO: Address API design, modularity, and extensibility.

\section{Chapter Summary}
\label{sec:demo:summary}
% TODO: Summarize the key takeaways from the demonstrations.        % Chapter 6: LoRA Limits (New chapter)
% \chapter{Application: Enhancing Comparative Question Answering with LLMs}
\label{ch:comparative_qa}

\section{Introduction}
\label{sec:compqa:intro}
% TODO: Introduce Comparative QA and the CAM 2.0 system.

\section{The CAM 2.0 System Architecture}
\label{sec:compqa:cam2_arch}
% TODO: Describe the overall architecture of CAM 2.0.

\section{LLM Integration and Fine-tuning}
\label{sec:compqa:llm_integration}
% TODO: Detail your contributions: LLM selection (Llama, Vicuna), fine-tuning process, integration challenges.

\section{Experimental Setup}
\label{sec:compqa:setup}
% TODO: Describe datasets, baselines, and evaluation metrics used for CAM 2.0.

\section{Results and Analysis}
\label{sec:compqa:results}
% TODO: Present the performance results, focusing on the impact of the LLM components you implemented.

\section{Discussion}
\label{sec:compqa:discussion}
% TODO: Discuss the effectiveness of LLMs in this complex QA task. Potential connections to KG methods.

\section{Chapter Summary}
\label{sec:compqa:summary}
% TODO: Summarize the findings regarding LLMs in comparative QA.  % Deleted - content integrated elsewhere
% \chapter{Integrated Evaluation and Comparative Analysis}
\label{ch:evaluation}

\section{Introduction}
\label{sec:evaluation:intro}
% TODO: Outline the goals of the integrated evaluation.

\section{Evaluation Framework}
\label{sec:evaluation:framework}
% TODO: Describe the benchmarks, metrics, and setup for comparison.

\section{Comparative Analysis: Core QA Methods}
\label{sec:evaluation:core_methods}
% TODO: Compare Reranking vs. ShortPathQA vs. Type Selection. Analyze trade-offs.

\section{Evaluation: LLMs in Comparative QA}
\label{sec:evaluation:comparative_qa}
% TODO: Present evaluation results for the LLM components in CAM 2.0.

\section{Overall Discussion}
\label{sec:evaluation:discussion}
% TODO: Synthesize the evaluation findings across all methods.

\section{Chapter Summary}
\label{sec:evaluation:summary}
% TODO: Summarize the main comparative insights.    % Deleted - content integrated elsewhere

% Note: chapter_3_subsection_type_selection.tex and chapter_3_subsection_kg_path_fusion.tex
% should be \input{}'ed inside chapter_3_methods.tex where appropriate.

\chapter{Conclusion and Future Work}
\label{chap:conclusion}

% 7.1 Synthesis of Contributions
\section{Synthesis of Contributions}
\label{sec:conclusion_synthesis}
% Comment: Provide a comprehensive summary of the thesis's achievements. Reiterate the 
% novel methods for KG-LLM fusion, the ShortPathQA benchmark contribution, system 
% demonstrations, and insights from related studies (LoRA). Emphasize the overall 
% narrative: identifying the LLM factuality problem and providing KG-based solutions.


% 7.2 Revisiting Research Questions
\section{Revisiting Research Questions}
\label{sec:conclusion_revisiting_rqs}
% Comment: Directly address the research questions posed in Chapter 1, providing 
% concise answers based on the evidence presented throughout the thesis.


% 7.3 Limitations of the Presented Work
\section{Limitations of the Presented Work}
\label{sec:conclusion_limitations}
% Comment: Honestly discuss the limitations of your research. This might include the 
% scope (e.g., focus on factoid QA), the specific KGs or LLMs used, scalability 
% challenges of the proposed methods, or aspects not fully explored.


% 7.4 Future Research Directions
\section{Future Research Directions}
\label{sec:conclusion_future_work}
% Comment: Propose concrete directions for future work building upon your thesis. 
% Examples: exploring more complex reasoning tasks, applying methods to different 
% domains or languages, integrating dynamic/temporal KGs, improving the efficiency 
% and scalability of fusion methods, developing more sophisticated control mechanisms, 
% investigating user interaction with KG-aware LLMs.       % Chapter 7: Conclusion (New chapter, replaces old conclusions.tex)
% \addcontentsline{toc}{chapter}{Conclusions}
\chapter*{Conclusion}


%1. Restate the Research Problem and Objectives
%Begin by briefly restating the research problem and the primary objectives of your thesis. This reminds readers of the central focus of your work.

% Example:
% "In this thesis, we addressed the challenge of integrating renewable energy sources into power systems, with a focus on improving reliability and optimizing power flow under uncertainty. Our primary objectives were to develop advanced methods for reliability assessment and to propose novel optimization techniques for chance-constrained optimal power flow."

In this thesis, we addressed the the challenges that occur in modeling power systems that experience high level of renewable energy penetration. Our main objective was to develop advanced statistical methods for current operating point reliability assessment and to propose novel optimization techniques for chance-constrained optimal power flow in various settings.

% 2. Summarize Key Findings
% Summarize the key findings and contributions of your research. Highlight the most important results and how they address the research problem.

% Example:
% "Our research led to the development of an adaptive importance sampling method, which significantly improves the accuracy and efficiency of risk estimation for reliability constraints. Additionally, the proposed A-priori Reduced Scenario Approximation (AR-SA) method reduces the number of samples required for reliable solutions in joint chance-constrained dynamic optimal power flow problems. These methods were validated through extensive simulations, demonstrating their effectiveness in handling uncertainties in power systems."

This research led to the development of adaptive importance sampling methods for grid reliability estimation and proposed new techniques for constructing scenario approximation for static and dynamic formulation of optimal power flow.

% 3. Discuss the Significance and Impact
% Discuss the broader significance and impact of your findings. Explain how your research contributes to the field and its potential real-world applications.

% Example:
% "The findings of this thesis have significant implications for the integration of renewable energy sources into power systems. By enhancing the reliability and efficiency of power system operations, our methods support the transition to sustainable energy solutions, contributing to global efforts to reduce greenhouse gas emissions and improve energy security. Furthermore, the proposed techniques can be applied to other areas of power systems engineering, offering a foundation for future advancements in the field."

The findings have significant implications on computing and estimating generation regimes of power grids, allowing for non-restrictive robust, statistical based calculation of generation regimes and efficient estimation of the latter's reliability. This contributes to safer transition to sustainable energy solutions, contributing to global efforts to reduce greenhouse gas emissions and improve energy security. Furthermore, the proposed techniques can be applied to other areas of power systems engineering and beyond, where uncertainty arises in optimization problems, offering a foundation for future advancements in the field.

% 4. Address Limitations
% Acknowledge any limitations of your research. Being transparent about the constraints and challenges you encountered adds credibility to your work.

% Example:
% "While our methods offer substantial improvements, there are limitations to consider. The adaptive importance sampling method relies on accurate physical information, which may not always be readily available. Additionally, the computational complexity of the AR-SA method, though reduced, may still pose challenges for extremely large-scale power systems."

Though the methods offer substantial improvements, there are limitations to consider. The method rely on high-voltage assumption, leading to a linear system/problems. However, the results can be generalized for non-linear cases by additional mathematical effort.

% 5. Suggest Future Work
% Suggest directions for future research based on your findings. Identify areas that require further investigation and how they can build on your work.

% Example:
% "Future research could explore the application of the adaptive importance sampling method to real-time power system operations, addressing the challenge of obtaining real-time physical data. Further development of the AR-SA method could focus on enhancing its scalability and applicability to even larger power grids. Additionally, integrating these methods with emerging technologies, such as smart grid systems and advanced forecasting techniques, presents a promising avenue for future work."

Future research could explore applications for non-linear settings, broader distribution class support and wider range of engineering applications since data-driven stochastic optimization is ubiquitous at this moment. The former could be achieved by iterative constructions of convex restrictions and the latter with distributionally robust optimization techniques. 

% 6. Final Thoughts
% Conclude with a few final thoughts that encapsulate the essence of your research and its potential to inspire further advancements in the field.

% Example:
% "In conclusion, this thesis contributes to the ongoing efforts to integrate renewable energy sources into power systems more effectively. The developed methods not only address current challenges but also pave the way for future innovations. As the global energy landscape continues to evolve, the insights gained from this research will be instrumental in shaping a sustainable and resilient energy future."

In conclusion, this thesis contributes to the ongoing efforts to integrate renewable energy sources into power systems more effectively. The developed methods not only address current challenges but also pave the way for future innovations. As the global energy landscape continues to evolve, the insights gained from this research will be instrumental in shaping a sustainable and resilient energy future.



% \newpage
% % \addcontentsline{toc}{chapter}{List of figures}
% \listoffigures

% % \addcontentsline{toc}{chapter}{List of tables}
% \listoftables      % Old conclusions include - replaced by chapter_7_conclusion

% \printnomenclature[3.5cm] % Значение ширины столбца с обозначениями стоит подбирать вручную


\addcontentsline{toc}{chapter}{List of symbols and abbreviations}
\chapter*{List of symbols and abbreviations}

\section*{Abbreviations and Acronyms}
\begin{tabularx}{\textwidth}{lX}
    ACT & Answer Candidate Type \\
    AI & Artificial Intelligence \\
    BART & Bidirectional and Auto-Regressive Transformers \\
    BERT & Bidirectional Encoder Representations from Transformers \\
    G2T & Graph2Text \\
    GAP & Graph-Aware Pre-training \\
    GCN & Graph Convolutional Network \\
    IR & Information Retrieval \\
    KG & Knowledge Graph \\
    KGQA & Knowledge Graph Question Answering \\
    LLM & Large Language Model \\
    LoRA & Low-rank Adaptation \\
    mGENRE & Multilingual Generative ENtity REtrieval \\
    MLKGQA & Multilingual Knowledge Graph Question Answering \\
    MPNet & Masked and Permuted Pre-training for Language Understanding \\
    MRR & Mean Reciprocal Rank \\
    NER & Named Entity Recognition \\
    NLP & Natural Language Processing \\
    PLM & Pre-trained Language Model \\
    QA & Question Answering \\
    RAG & Retrieval-Augmented Generation \\
    RMSE & Root Mean Square Error \\
    SP & Semantic Parsing \\
    SSM & Salient Span Masking \\
    SVG & Scalable Vector Graphics \\
    T5 & Text-To-Text Transfer Transformer \\
\end{tabularx}

\section*{Mathematical Symbols}
\begin{tabularx}{\textwidth}{lX}
    $C$ & Candidate list or set of answer candidates \\
    $c^*$ & Correct answer for a given question \\
    $c_1$ & Top-ranked candidate answer \\
    $d$ & Density parameter in graph density formula; damping factor in PageRank \\
    $G$ & Graph or subgraph \\
    $I$ & Identity matrix \\
    $\lambda$ & Eigenvalue; diversity penalty parameter in diverse beam search \\
    $m$ & Number of edges in a graph \\
    $n$ & Number of nodes in a graph \\
    $PR(p_i)$ & PageRank of Graph $G$ node $p_i$ \\
    $Q$ & Set of all questions in evaluation \\
    $\mathcal{A}_q$ & Set of answer candidates for question $q$ \\
    $\mathcal{E}_q$ & Set of entities mentioned in question $q$ \\
    $\mathcal{G}(q, c)$ & Subgraph connecting question $q$ entities to candidate $c$ \\
    $\mathcal{L}(q, c)$ & Linearization of subgraph $\mathcal{G}(q, c)$ \\
    $\mathcal{N}(\mu, \Sigma)$ & Gaussian distribution with mean vector $\mu$ and covariance matrix $\Sigma$\\
    $\mathcal{N}_q$ & Set of negative (incorrect) candidates for question $q$ \\
    $1\left[A\right]$ & Indicator of the event $A$ \\
    $\mathbb{I}(\cdot)$ & Indicator function that returns 1 if condition is true, 0 otherwise \\
\end{tabularx}        % List of Acronyms
% \include{Dissertation/dictionary}      % Dictionary of Terms (Optional)
\include{Dissertation/references}      % Bibliography
\include{Dissertation/lists}           % List of Tables and Figures

\setcounter{totalchapter}{\value{chapter}} % Count chapters

%%% Appendices Setup
\appendix
% Оформление заголовков приложений ближе к ГОСТ:
\setlength{\midchapskip}{20pt}
\renewcommand*{\afterchapternum}{\par\nobreak\vskip \midchapskip}

\ifnumequal{\value{englishthesis}}{0}{
    \renewcommand\thechapter{\Asbuk{chapter}} % Чтобы приложения русскими буквами нумеровались
}{}

% \include{Dissertation/appendix}        % Приложения

\setcounter{totalappendix}{\value{chapter}} % Подсчёт количества приложений

\end{document}
